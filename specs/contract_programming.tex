\documentclass{report}
\title{Felix Assertions}
\author{John Skaller}
\begin{document}
\maketitle
\tableofcontents

\chapter{Overview}
Felix language provides syntax to specify axioms, lemmas, and theorems
and other semantic properties of functions, types and program state. This semantic
information can be checked in two ways: by invoking offline theorem
provers on generated files containing axioms and proof obligations,
and by run time testing. The compiler can also use special logic
constructs called reductions to optimise evaluation.

It should be emphasised, however, that the primary intent of logic
elements in Felix is not for proving or checking correctness
mechanically, but simply to provide a syntax for {\em stating}
semantics so as to enable programmers to better understand
the intent, operation, and constraints on use of programs
and libraries.

The facilities provided are rudimentary, that is, they're
quite weak, compared to those available in proof assistants,
however the mere existence of these facilities is a major
advance over most other programming languages, including
advanced functional languages such as Ocaml and Haskell.

\section{Axioms, Lemmas and Theorems}
Felix provides constructions to state axioms, lemmas,
and theorems with equational first order predicates and universal
head quantification. Existential quantifiers are not
supported, and universal quantification is represented
by variables which scope over whole expressions.

Polymorphic logic elements are allowed, and have two meanings.
Outside of typeclasses, they're universal. Such axioms can
only be used to prove properties of polymorphic functions
or categorical constructions such as tuple formation.

Inside typeclasses, axioms also act as constraints on 
type class instance implementations. For example if
a type-class function is associative, all instances
of it must be associative. Thus, such axioms generate
obligations on the instantiator, and in effect become
theorems of the instance which require proof.

Theorems inside typeclasses should be provable from
the axioms for all instances, and if obligations
are generated if the theorem is specified as a lemma.

An axiom is a basic assertion. A lemma is an assertion
which is considered simple enough to be proved automatically
from axioms by a reasonably competent automatic theorem prover.
A theorem is an assertion which is expected to be hard enough to
prove that it requires a proof outline and/or hints to a
proof assistant as to what strategy to use to find a proof.
There is currently no support for providing such assistance.

Felix emits data relating to types and functions, together
with axioms and lemmas in the \verb%why% format,
which allows either the \verb%ergo% theorem prover to attempt
a proof directly, or the \verb%why% program to translate
the specifications to a format supported by one of several
supported theorem provers, such as \verb%simplify%.

Because theorem proving is slow, and proof assistants require
considerable expertise to use, Felix does not invoke any provers
or assistants automatically, however specifications are generated
in \verb%*.why% files automatically.

Axioms, lemmas and theorems are stated like polymorphic functions:
\begin{verbatim}
axiom symmetry[v] (a:v, b:v) f(a,b) == f(b,a);
lemma add0 (a:int) a + 0 == a;
\end{verbatim}

\subsection{Default Virtuals}
In addition to explicit constraints, a typeclass may
provide a default implementation for a virtual function.
\begin{verbatim}
virtual fun plus: v * v -> v;
virtual fun neg: v -> v;
virtual fun sub (a:v, b:v) => plus (a, neg b);
\end{verbatim}

Such implementations generate an implementation in an
instance if one is not provided. If an implementation is
provided in an instance, it must be a more efficient implementation
of the function; that is, it must conform to the semantics
implied by the default. The compiler is free to reject a program
if it is able to prove this is not the case.

Default virtuals can also provide another more subtle 
completeness obligation on a typeclass instance breach of which 
which may not be detected by the compiler: the breaking of
circular definitions. In this example:
\begin{verbatim}
virtual fun neg (a:v) => sub (zero(), a);
virtual fun sub (a:v, b:v) => plus (a, neg b);
\end{verbatim}

instances much provide at least one of the functions
to break the circular definition, otherwise an infinite
recursion will occur at run time. Although circularity can
be detected by the compiler, it cannot easily distinguish
unconditional circularity as above, and conditional
circularity in which a branch might terminate a recursion.

In summary we may say that an instance is obliged to 
be complete; that is, to implement a basis, and, it is
obliged to be conforming, that is, to behave according
to the explicitly specified axioms and those implied by
default implementations.

\section{Reductions}
Another form of logic is the reduction:
\begin{verbatim}
reduce add0 (a:int) a + 0 => a;
\end{verbatim}
In this form, a logic element actually effects the replacement
of expressions of form $a + 0$ where a is any \verb%int% expression
with just $a$. A real example is that revering a list twice leaves
the original list: although programmers rarely reverse lists twice,
this can easily happen if when functions are inlined, and so the
optimisation provided can be quite useful.

Search for patterns matching the head of a reduction is computationally
expensive: at least quadratic since every reduction is matched against
every subexpression in the program.

Therefore Felix does these searches in a limited number
of places, so there is no assurance a reduction which could be applied
at some stage actually will be. The system uses a garbage collection
concept to discard axioms applying to types which are never used, to
avoid fruitless match attempts.

Note: there is currently no way to make assertions about procedures, the system
only works for expressions.

\section{ad hoc assertion}
Felix provides the usual run time checking with 
\begin{verbatim}
assert condition;
\end{verbatim}
which generates a run time check which aborts the program
if the condition evaluates false. Assertions cannot be disabled,
they're always checked.

\section{Pre and post conditions}
For functions pre and
post conditions can also be given:
\begin{verbatim}
fun f(a:int where a>0):int expect result >0 => a;
\end{verbatim}
A precondition is specified by a \verb%where% clause
after all the function arguments, a postcondition is
specified by an \verb%expect% clause, after the return type
is specified, but before the body is given. The special identifier
\verb%result% can be used to indicate the function return value.

Pre- and post- conditions usually generate run time checks.
However the compiler is free to attempt to prove the post-
condition from the pre-condition or the actual program code
and reduce or elide it.

\section{Axiom Checks}
In addition, axioms can be checked with a \verb%check_axiom%
directive. This looks like:
\begin{verbatim}
check_axiom(2,3);
\end{verbatim}
This example checks the expression of all axioms which can take
two integer arguments, and aborts with a diagnostic if any
axiom fails. However, if an axiom has a pre-condition, then
the check is not attempted if the pre-condition fails: in this
case the axiom doesn't apply to the data.

\end{document}

