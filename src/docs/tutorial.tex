
\documentclass[oneside]{book}
\usepackage{color}
\definecolor{bg}{rgb}{0.95,0.95,0.95}
\usepackage{parskip}
\usepackage{minted}
\usemintedstyle{friendly}
\setminted{bgcolor=bg,xleftmargin=20pt}
\usepackage{hyperref}
\hypersetup{pdftex,colorlinks=true,allcolors=blue}
\usepackage{hypcap}
\title{Introduction to Felix}
\author{John Skaller}
\begin{document}
\maketitle
\tableofcontents
\chapter{Quick Start}
We must of course begin with the traditional greeting!

\begin{minted}{felix}
println$ "Hello World";
\end{minted}

Here \verb%println% is a procedure which outputs a value to
standard output. The argument is of course literal of
type \verb%string%. Calls to procedures must be terminated
by a semicolon \verb%;%. You will note, we do not require
parentheses around the argument to denote a procedure call!
We do not like parentheses much! The \verb%$% sign will be
explained in more detail later, but for now you should
know it is just a low precedence, right associative application
or call operator.

You can run this program from your console or terminal, once
Felix is installed, by just typing:

\begin{verbatim}
flx hello.flx
\end{verbatim}

assuming the file \verb%hello.flx% contains the sample code
and is in the current directory. Behind the scenes Felix does
dependency checking, transates the program to C++, compiles
the C++ to a machine binary, and runs it.

It works like Python but it performs like C++.

For more information see \url{http://felix-lang.org}.

\section{Integers: {\tt int}}
Felix has the usual integer type \verb%int% and literals consisting
of decimal digits, and the usual binary operators \verb%+% for addition,
\verb%-% for subtraction, \verb%*% for multiplication, \verb%/% for division,
and \verb+%+ for remainder. These have the same semantics as in C, because,
these operators are in fact implemented in the Felix standard library by
delegating to C.

\begin{minted}{felix}
println$ 42;
println$ (42 + 12) * 90 - (36 / 2 + 1) * 127 % 3;
\end{minted}

Note you must put spaces around the \verb%-% operator! This is because
in Felix \verb%-% is also a hyphen, allowed in identifier names.

We also have the unary operator \verb%-% for negation, and for symmetry
we also have unary \verb%+% which does nothing.

These operators are just functions so here are some more: 
the function \verb%str% converts an \verb%int% to a readable
string, the function \verb%abs% finds the absolute value of an
integer and \verb%sgn% returns $-1$ if its argument is negative,
$0$ if it is zero, and $1$ if it is positive.

We also have the usual comparisons on integers represented as
infix operators: \verb%==% for equality, \verb%!=% for inequality, 
\verb%<% for less than, \verb%>% for greater than, and
\verb%<=% and \verb%>=% for less than or equal and greater than
or equal, respectively.

\section{Booleans: {\tt bool}}
The result of a comparison is a new type, \verb%bool%
which has two values, \verb%false% and \verb%true%.
You can print booleans:

\begin{minted}{felix}
println$ 1 < 2;
\end{minted}

Felix also has a special statement for asserting that a 
boolean value is true:

\begin{minted}{felix}
assert$ 1 < 2;
\end{minted}

If the argument of an \verb%assert% is false, then if control
flows through it, the program is terminated with an error message.

Booleans support the usual logical operators, but in Felix they
are spelled out. Conjunction is spelled \verb%and%, whilst disjunction
is spelled \verb%or%, implication is spelled \verb%implies%. Of course
we also have negation \verb%not%.

\subsection{Conditional Expression}
With \verb%bool% and \verb%int% we can demonstrate the conditional
expression:

\begin{minted}{felix}
println$ 
  if 1 < 2 then "less" 
  elif 1 > 2 then "greater" 
  else "equal" 
  endif
;
\end{minted}

Note that bool also forms a total order and can be compared, we
have $\verb%false% < \verb%true%$ so that $a \verb%==% b$
means $a$ is equivalent to $b$, we can also say that $a$ is true
if and only if $b$ is true. It turns out inequality $a \verb%!=% b$
is the same as exclusive or. Be careful though, since if
$a$ is less than or equal to $b$, written $a \verb%<=% b$ this actually
means that $a$ implies $b$ logically!

\section{Variables: {\tt var}}
Felix is a procedural programming language, so it has variables!
A variable denotes an addressable, mutable, storage location
which in Felix, like C, is called an {\em object}.

\begin{minted}{felix}
var x = 1;
var y = 2;
var z = x + 2 * y;
println$ z;
z = 2 * x + y;
println$ z;
\end{minted}

This code shows variables can be used to factor expressions into
a sequence of assignments. We assign variable $x$ the value 1,
variable $y$ the value 2, add $x$ to twice $y$ and put the
result in variable $z$, then print it.

Then we assign perform a different calculation and assign
that value to $z$ and print it.

What appears to be a variable initialisation is actually
equivalent to a definition of an uninitialised variable
followed by an assignment.

\begin{minted}{felix}
var x: int;
x = 1;
\end{minted}

The first statment reserves uninitialised store of type \verb%int%
named $x$ and the second assigns a value to it. Be very careful
with variables, initialised or not!  Felix has setwise scoping
rules, which are similar to C's function scope used for labels.
This means in a scope, you can refer to any symbol defined
{\em anywhere} in that scope. We shall see this is useful
for recursive functions because it eliminates the need for
forward declarations. However the following code has undefined behaviour:

\begin{minted}{felix}
println$ x;
var x = 1;
\end{minted}

There is no syntax error, no type error, and no lookup error in 
this code. The programmer used an uninitialised variable:
even though the variable is assigned a value, it is done too late.

Strangely, this code has deterministic behaviour:

\begin{minted}{felix}
println$ x;
var x = "Hello";
\end{minted}

but it may not do what you expect! It prints nothing!
The reason is simple enough: when Felix creates a variable
it is first initialised with its C++ default constructor.
Since a Felix \verb%int% is literally a C++ \verb%int% the
default constructor exists, but it is said to be trivial,
meaning, it does nothing. This is to improve performance,
in the case the first use of the variable will be to assign
a value to it: there's no point putting a value in there and
then overwriting it!

On the other hand Felix \verb%string% type is just 
C++ \verb%::std::basic_string<char>% and its default 
initialiser sets the string to the empty string \verb%""%.
That's what the code above prints!

\section{Floating Point: {\tt double}}
Felix also provides a model of C++ type \verb%double%
with the usual operators.  This is a double precision floating point type 
which usually follows IEEE standard. You can write a double precision
literal in the usual way. Felix follows ISO C-99 for floating point
literals.

A set of useful functions is also provided, corresponding to those
found in C-99 header file \verb%math.h%.

\begin{minted}{felix}
var x = 1.3;
var y = 0.7;
assert sqrt (sqr (sin x) + sqr (cos y)) - 1.0 < 1E-6;
\end{minted}

Note there is a special caveat with floating poimt arithmetic.
In Felix, \verb%-% has higher precedence than \verb%+%.
This means that:

\begin{minted}{felix}
var x: double = something;
var y: double; something_else;
assert x + y - y == 0.0;
\end{minted}

because the subtraction is done first.  This can make
a difference for integers too, if a calculation overflows,
but most floating point types are not associative: order matters!

Similarly, division has a higher precedence than multiplication!

\section{Strings: {\tt string}}
Felix uses C++ strings for its own strings for compatibility.
String literals have 6 forms following Python. Strings
not spanning multiples lines can be enclosed in either single
or double quotes. Strings spanning multiple lines may
be enclosed in tripled single or double quotes.

\begin{minted}{felix}
var ss1 = 'Short String';
var ss2 = "Short String";
var ls3 = """
A poem may contain
many lines of prose
""";
var ls4 = '''
Especially if it is written
by T.S. Elliot
''';
\end{minted}

Note that the triple quoted strings contain everything between the
triple quotes, including leading and trailing newlines if present.

Strings can be concatenated by writing them one after the other
separated by whitespace.

\begin{minted}{felix}
var rose = "Rose";
var ss5 = 
  "A " rose ", "
  "by another "
  "name."
;
\end{minted}

Note that concatenation works for string expressions in general,
not just literals.

\subsection{Escape Codes}
Special escapes may be included in strings. The simple
escapes are for newline, \verb%\n%, tab \verb%t%,
form feed \verb%\f%, vertical tab \verb%\v%, 
the escape character \verb%\e%, \verb%\a% alert
or bell, \verb%\b% backspace, \verb%\'% single quote,
\verb%\"% double quote, \verb%\r% carriage return,
\verb%\\% backslash (slosh).

These can be used in any simple string form.
Note carefully each is replaced by a single character.
This includes \verb%\n%, even on Windows.

In addition Felix provides \verb%\xXX% where each
X is one of the hex digits 0123456789ABCDEFabcdef.
The hex escape is at most two characters
after the \verb%x%, if the second character is not a
hex digit, the escape is only one character long,
the sequence is replaced by the char with ordinal value 
given by the hex code.

Felix also provides decimal and octal escapes using
\verb%\dDDD% and \verb%\oOOO% respectively, with a 
3 character limit on the decoder. Note carefully
Felix does NOT provide C's octal escape using
a 0 character. Octal is totally archaic. 

Felix also provides two unicode escapes. These
are \verb%\uXXXX% and \verb%\UXXXXXXXX% which 
consist of up to 4 and up to 8 hex digits exactly. 
The corresponding value is translated to UTF-8
and that sequence of characters replaces the escape.
The value must be in the range supported by UTF-8.

Felix also provides raw strings, in which escapes
are not recognised. This consists of the letter \verb%r%
or \verb%R% followed by a single or triple quoted string
with double quote delimiter. You cannot use the raw prefix
with single quoted strings because the single quote following
a letter is allowed in identifiers.

\subsection{String Functions}
Felix has a rich set of string functions. The most important
is \verb%char%, which returns a value of type \verb%char%.
If the string argument has zero length, the character with
ordinal value 0 is returned, otherwise the first character
of the string is returned. Again, following Python, Felix does 
not provide any character literals!

\subsubsection{Comparisons}
We provide the usual comparison operators.

\subsubsection{Length}
Felix also provides the most important function \verb%len%
which returns the length of a string. The return type is
actually \verb%size% which is a special unsigned integer type
corresponding to ISO C's \verb%size_t%.

\subsubsection{Substring}
Felix can fetch a substring of a string using Python like convenions.

\begin{minted}{felix}
var x = "Hello World";
var copied = x.[to];   // substring
var hello = x.[to 5];  // copyto
var world= x.[6 to];   // copyfrom
var ello = x.[1 to 5]; // substring
var last3 = x.[-3 to]; // substring
\end{minted}

The first index is inclusive, the second exclusive. 
The default first position is 0, the default last
position is the length of the string. If the range specified
goes off either end of the string it is clipped back to the string.
If the indices are out of order an empty string is returned.

A negative index is translated to by adding the string length.

The substring function s defined so it cannot fail.
The name of the actual library function called by this notation
is shown in the corresponding comment.

\subsubsection{Index}
To fetch a single character use:

\begin{minted}{felix}
var x = "Hello world";
var y : char = x.[1]; // subscript
\end{minted}

If the index is out of range, a character with ordinal 0 is returned.
Negative indices are translated by adding the string length.
The index function cannot fail.

\section{Tuples}
Felix has an strucurally typed product where components are
accessed by position, commonly called a {\em tuple}.  Tuples
can be constructed using the non-associative n-ary comma operator
and accessed by using a plain decimal integer as the projection
function:

\begin{minted}{felix}
var x = 1,"Hello", 42.7; // type int * string * double
var i = x . 0;
var h  = x . 1;
var d = x  . 2;
\end{minted}

Tuple is just another name for Cartesian product.
They allow you to pack several values together into a single
value in such a way that you can get the components you put
in out again.

We shall see tuples are vital for functions, since functions
can only take a single argument. To work around this fact,
we can pack multiple values together using a tuple.

\section{Lists}
A list is a variable length sequence of values of the same type.
An empty list of \verb%int% is denoted \verb%Empty[int]%.
Given a list you can create a new one with a new element
on the front using the constructor \verb%Cons% as follows:

\begin{minted}{felix}
var x = Empty[int];
x = Cons (1,x);
x = Cons (2,x);
x = Cons (3,x);
var y = Cons (3, Cons (2, Cons (1, Empty[int])));
\end{minted}

Of course this is messy! Here is a better way:

\begin{minted}{felix}
var z = list (3,2,1);
\end{minted}

This method converts a tuple to a list. You can add two lists
together, prepend a value, or add a value to the end of a list
with the infix \verb%+% operator:

\begin{minted}{felix}
var x = list (1,2,3);
x = 1 + x + x + 42;
\end{minted}

Take care that + associates to the left and you don't
accidentally add two integers together! There is a second
operator you can use as well which is right associative and
prepends an element to a list:

\begin{minted}{felix}
var x = 3 ! 2 ! 1 ! Empty[int];
x = 42 ! x;
\end{minted}

You can use the \verb%len% function to find the length of a list,
and test if an element is in a list using the \verb%in% operator:

\begin{minted}{felix}
var x = list (1,2,3);
assert len x in x; // 3 is in the list!
\end{minted}

Lists in Felix are purely functional data structures: you cannot
modify a list. All the nodes in a list are immutable, which
means when you prepend an element $A$ to a list $L$, and then
prepends an element $P$ to the same list $L$, the tail of the list
is shared. Lists can be passed around efficiently without copying.

When some prefix of a list is no longer accessible because the function
prepending the prefix returns without saving the list, the prefix
elements will be removed automatically by the Felix garbage collector.

A list can be taken apart with a pattern match:

\begin{minted}{felix}
var x = list (3,2,1);
println$
  match x with
  | #Empty =>  "Empty";
  | Cons(v, tail) => "first element " + v.str;
  endmatch
;
\end{minted}

\section{Functions}
We have enough preliminaries now to finally introduce functions.
Without further ado, here are some basic functions:

\begin{minted}{felix}
fun twice (x:int) => x + x;
fun thrice (x:int) => twice x + x;
\end{minted}

Functions also have a more expanded form:

\begin{minted}{felix}
fun trickdiv (num:int, denom:int) :int = 
{
   var y = if denom == 0 then 1 else denom endif;
   return xnum / y;
}
\end{minted}



\section{Procedures}
Stuff.

\chapter{Functional Programming}
We shall begin our more serious exploration of Felix with functional programming
techniques.


\begin{minted}{felix}
fun inner_strlist (x: list[int]) =>
  match x with
  // Emnpty list
  | #Empty => ""

  // One element list
  | Cons (head, #Empty) => str head 

  // Two element list
  | Cons (head, Cons (second, #Empty)) => str head ", " + str second 

  // More that two elements
  | Cons (head, tail) => str head ", " + inner_strlist tail
  endmatch
;

fun strlist (x:list[int]) => "list (" + inner_strlist x + ")";

println$ strlist (list (1,2,3));
\end{minted}

Here, the function \verb%inner_strlist% is recursive. It matches
the list so it is empty it produces an empty string.
If there is only one element, it produces the string 
representation of that int.

If the list contains two elements, it produces a string containing
the string representation of the two integers separated by a comma.
Otherwise it produces a string representation of the head, followed
by a comma, followed by a string representation of the tail.
The tail must contain at least two elements, otherwise the pattern
match would not reach that case.

However his function is not tail recursive! Lets rewrite it so 
it is:


\begin{minted}{felix}
fun inner_strlist (x: list[int], result:string) =>
  match x with
  // Emnpty list
  | #Empty => ""

  // One element list
  | Cons (head, #Empty) => str head 

  // Two element list
  | Cons (head, Cons (second, #Empty)) => str head ", " + str second 

  // More that two elements
  | Cons (head, tail) => inner_strlist (tail, str head ", ")
  endmatch
;
\end{minted}

A {\em tail call} is the final call in a function. Tail calls can
be optimised from a call followed by a return to just a jump. 
This avoids pushing the return address, then returning to that
address from the called procedure, then immediately returning
to the callers return address. Instead, the called procedure
just returns to the callers return address directly.

In particular, if the tail call is call to self, then by assigning
the call argument to the parameter, we then just jump to the start
of the function. In other words, we can optimise tail recursion
to a simple loop. This saves more than the space of the return
address on the stack, it saves the space required by all the local
variables as well. We can reuse these variables, since the old
values cannot be accessed.

Here is another, simpler tail recursive function:

\begin{minted}{felix}
fun addup (x: list[int]) =
{
  fun aux (rest: list, acc: int) =>
    match rest with
    | #Empty => rest
    | Cons (head, tail) => aux (tail, acc + head)
    endmatch
  ;
  return aux x 0;
}
\end{minted}

The non-recursive function \verb%addup% contains a nested tail recursive
function \verb%aux%.

Ok so, what if I wanted to add up the squares of the numbers 
in the list? Is there a way to avoid that?

The answer is yes:


\begin{minted}{felix}
fun fold(binop: int * int -> int, init:int, x:list[int]) =
{
  fun aux (rest: list, acc: int) =>
    match rest with
    | #Empty => rest
    | Cons (head, tail) => aux (tail, binop (acc,head))
    endmatch
  ;
  return aux x init;
}
\end{minted}

Fold works with any binary operator on \verb%int%. For example:

\begin{minted}{felix}
fun addition (acc:int,y:elt) => x + y;
assert addup x == fold (addition, 0, x);
\end(minted}

There is another way to write the folding expression:

\begin{minted}{felix}
assert addup x == fold (fun (acc:int, y:elt)=>x+y), 0, x);
\end{minted}

using an unnamed, or anonymous function, sometimes called 
a lambda.


\end{document}
