
\documentclass{article}
\usepackage{color}
\definecolor{bg}{rgb}{0.95,0.95,0.95}
\definecolor{emphcolor}{rgb}{0.5,0.0,0.0}
\newcommand{\empha}{\bf\color{emphcolor}}
\usepackage{parskip}
\usepackage{minted}
\usepackage{caption}
\usepackage{amsmath}
\usepackage{amssymb}
\usepackage{amscd}
\usepackage{makeidx}
\makeindex
\usemintedstyle{friendly}
\setminted{bgcolor=bg,xleftmargin=20pt}
\usepackage{hyperref}
\hypersetup{pdftex,colorlinks=true,allcolors=blue}
\usepackage{hypcap}
\newcommand*{\fullref}[1]{\hyperref[{#1}]{\autoref*{#1} \nameref*{#1}}}
\DeclareMathOperator{\quot}{div}
\DeclareMathOperator{\rmd}{rmd}
\title{Compact Linear Types}
\author{John Skaller}
\begin{document}
\maketitle
\section{Introduction}
Let us observe that the number of pennies in 3 pounds, 10 shillings, and 8 
pennies, is given by:
$$8 + 10 \times 20 + 3 \times 20 \times 12$$
because there are 12 pennies in a shilling, and 20 shillings in a pound.
Generalising this idea we have:
$$a=x_0+\sum_{i=1}^{n-1} (x_i \prod^{i-1}_{j=0} c_j)$$
which is a general finite positional number system, with $j$ ranging from 0 to $n-2$ 
representing an n position system, and $c_j$ representing the number of digits
in the j'th position.

Decoding is given by 
$$x_i = (a \rmd \prod^{n-2}_{j=i} c_j) \quot \prod^{i-1}_{j=0} c_j$$

\section{Application to types}
Let the symbol decimal integer $n$ represent the sum of $n$ units,
where the unit type is the type of an empty tuple. Call the
type void, the sum of no units, and observe

\begin{minted}{felix}
typedef void = 0;
typedef unit = 1;
typedef bool = 2;
\end{minted}

Call any such type a unitsum. Now observe that the positional number
system described in section 1 may be used to encode and decode 
values of a finite product of unit sums by adding a limit on the
number of digits in the higest term.

The formula for the $x_i$ are then projections of the product
which is formed by the formula for the $a$. The set of integers
calculated by the constructor then has size
$$N=\prod_{j=0}^{n-1}c_j$$
and the set is compact and consists of the values 0 through $N-1$
and is therefore also linear.

The system is also familiar as the decomposition of cyclic groups
and exhibits the isomorphism
$$Z_N \cong {\mathbb Z}_{j_0} \times {\mathbb Z}_{j_1} \ldots {\mathbb Z}_{j_{n-1}}$$

\section{Application to arrays}
A fundamental problem in computing is that a loop
through an array of arrays requires two loops:

\begin{minted}{felix}
  var a = ((1,2,3),(4,5,6));
  var sum = 0;
  for i in 0..<2 
    for j in 0..<3 perform 
       sum += (a.i).j;
  println$ sum;
\end{minted}

and for an array of rank $k$ we need $k$ loops. There is a two 
step solution to this problem. We desire to make the transformation
of types

$${{\tt int}^3}^2 \longrightarrow 
{\tt int} ^ {3 \times 2}$$

This has the effect of changing the data functor to a single
array, indexed by a tuple:

\begin{minted}{felix}
  var a = ((1,2,3),(4,5,6));
  var b = a :>> (int^(2 * 3));
  var sum = 0;
  for i in 0..<2
    for j in 0..<3 perform
       sum += b.(i:>>2,j:>>3);
  println$ sum;
\end{minted}

Although we still need two indices and two loops, we have
formed a new data functor by composition of two array functors.

A second isomorphism solves our problem:

$$ {\tt int} ^ {2 \times 3} \longrightarrow {\tt int} ^ 6 $$

with

\begin{minted}{felix}
  var a = ((1,2,3),(4,5,6));
  var b = a :>> int^(2*3);
  var c = b :>> int^6;
  var sum = 0;
  for i in 0..<6 perform
    sum += c.i;
  println$ sum;
\end{minted}

performing a compact linear encoding.

\section{Extension to Arrays}
If the base type of an array is itself compact linear,
then the whole array is also compact linear. For example

\begin{minted}{felix}
var x : bool ^ 4 = true, false, true, true;
println$ x :>> int; 
\end{minted}

Bit arrays drop out of compact linear types as a special case.

\section{Extension to sums}
We consider first the case of two arrays of the
same base type. If the first has length $n$ and the second length $m$
then an index of type $n+m$ is interpreted as follows: first chose
which array to index, then use the argument of the selected case
as the index. By concatenating the arrays, we see the index is
isomorphic to an array of length $n+m$.

This suggests a sum of compact linear types is also compact linear.

The size of a compact linear type is given inductively by
the rule:

\begin{itemize}
\item the size of a unitsum $n$ is $n$
\item the size of a product of compact linear types is the product
of the sizes
\item the size of an exponential of compact linear types is
the exponential of the sizes
\item the size of a sum of compact linear types is the
sum of the sizes
\end{itemize}

The encoding formula is:
$$a = y_i \sum^{i-1}_{j=0} z_j$$
where $z_j$ is the size of the j'th case. The $y_i$ are readily
identified as the characteristic injections of the sum type.

The decoding formula is in two parts: we must first calculate
the correct case index, it is the unique $j$ such that 
$$z_{j} > a \ge z_{j-1}$$
Then the argument is given by
$$y_j = a - \sum_{i=0}^{j-1}z_i$$

\end{document}
