\documentclass{article}
\usepackage{color}
\definecolor{bg}{rgb}{0.95,0.95,0.95}
\definecolor{emphcolor}{rgb}{0.5,0.0,0.0}
\newcommand{\empha}{\bf\color{emphcolor}}
\usepackage{parskip}
\usepackage{minted}
\usepackage{caption}
\usepackage{amsmath}
\usepackage{amssymb}
\usepackage{amscd}
\usepackage{makeidx}
\makeindex
\usemintedstyle{friendly}
\setminted{bgcolor=bg,xleftmargin=20pt}
\usepackage{hyperref}
\hypersetup{pdftex,colorlinks=true,allcolors=blue}
\usepackage{hypcap}
\newcommand*{\fullref}[1]{\hyperref[{#1}]{\autoref*{#1} \nameref*{#1}}}
\DeclareMathOperator{\quot}{div}
\DeclareMathOperator{\rmd}{rmd}
\title{Exchange of Control}
\author{John Skaller}
\begin{document}
\maketitle
\section{Objects}
A coroutine system consists of the following types of objects:
\begin{description}
\item[Scheduler] A device to hold a set of active fibres and
select one to be current.
\item[Channels] An object to support synchronisation and data transfer.
\item[Fibres] A thread of control which can be suspended and resumed.
\item[Continuations] An object representing the future of a coroutine.
\end{description}

\subsection{Fibre States}
Each fibre is in one of these states:
\begin{description}
\item[Running] Exactly one fibre (per pthread) is always running.
\item[Active] Fibes which are ready to run but not running.
\item[Hungry] Fibres suspended waiting for input on a channel.
\item[Blocked] Fibres suspended waiting to perform output on a channel.
\item[Dead] A fibre for which there is no longer a current continuation.
\end{description}

\subsection{Channel States}
Each channel is in one of these states:
\begin{description}
\item[Empty] There are no fibres associated with the channel.
\item[Hungry] A set of hungry fibres are waiting for input on the channel.
\item[Blocked] A set of blocked fibres are waiting to perform output on the channel.
\end{description}

\section{Abstract Operations}
Several operations are specific to a fibrated system.
\subsection{Spawn} 
The spawn operation takes as an argument a unit procedure and makes
a closure thereof the initial continuation
of a new fibre.  Of the pair consisting of the currently running
fibre (the spawner) and the new fibre (the spawnee) one will have Active
state and the other will be Running. It is not specified which
of the pair is Running.

\subsection{Create channel}
A function which creates a channel.

\subsection{Read}
The read operation takes as an argument a channel.

\begin{enumerate}
\item If the channel is Empty, the Running fibre performing the read
changes state to Hungry, the channel changes state to Hungry,
and the fibre is associated with the channel.

If there are no active fibres, the program terminates,
otherwise the scheduler selects an Active fibre and
changes its state to Running.  It is not specified which active 
fibre is chosen.

\item If the channel is Hungry, the Running fibre changes state
to Hungry, and the fibre is associated with the channel.

If there are no active fibres, the program terminates,
otherwise the scheduler selects an Active fibre and
changes its state to Running.  It is not specified which active 
fibre is chosen.

\item If the channel is Blocked, one of the associated Blocked fibres
is selected, and dissociated from the channel. Of these two
fibres, one is changed to state Active and the other to Running.
It is not specified which fibre is chosen to be Running.

The value supplied to the write operation of the Blocked
fibre will be pass to the Hungry fibre when it transitions
to Running state.
\end{enumerate}


\subsection{Write}
The write operation takes two arguments, a channel and a value
to be written.

\begin{enumerate}
\item If the channel is Empty, the Running fibre performing the write
changes state to Blocked, the channel changes state to Blocked,
and the fibre is associated with the channel.

If there are no active fibres, the program terminates,
otherwise the scheduler selects an Active fibre and
changes its state to Running.  It is not specified which active 
fibre is chosen.

\item If the channel is Blocked, the Running fibre changes state
to Blocked, and the fibre is associated with the channel.

If there are no active fibres, the program terminates,
otherwise the scheduler selects an Active fibre and
changes its state to Running.  It is not specified which active 
fibre is chosen.

\item If the channel is Hungry, one of the associated Hungry fibres
is selected, and dissociated from the channel. Of these two
fibres, one is changed to state Active and the other to Running.
It is not specified which fibre is chosen to be Running.

The value supplied by the write operation of the Blocked
fibre will be pass to the Hungry fibre when it transitions
to Running state.
\end{enumerate}

\subsection{Reachability}
The Running, and, each Active fibre and its associated call chain of 
continuations are deemed to be Reachable.

If a channel is known to reachable fibre, it is also reachable.

Each fibre associated with a reachable channel is reachable.

The transitive closure of the reachability relation consists
of a closed, finite, collection or channels and fibres which
are reachable.

Unreachable fibres and channels are automtically garbage
collected.

\subsection{Elimination}
Fibres and channels are eliminated when they are 
no longer reachable.

A fibre may become unreachable in three ways.

\subsubsection{Suicide}
A fibre for the the initial continuation returns acquires
the Dead state. If there are no longer any Active fibres,
the program returns, otherwise the scheduler picks
one Active fibre and changes its state to Running.

A fibre in the Dead state should eventually become unreachable
if it is not unreachable when it becomes Dead.

\subsubsection{Starvation}
A fibre in the Hungry state may become unreachable,
in which case it is marked Dead and is no longer
part of the fibration system.

\subsubsection{Blockage}
A fibre in the Blocked state may become unreachable 
in which case it is marked Dead and is no longer
part of the fibration system.

\section{LiveLock}
If a fibre is Hungry (or Blocked) on a reachable channel
but no future Running fibre will write (or read) that
channel, the fibre is said to be livelocked. The fibre
will never proceed but it cannot be removed from
the system because it is reachable via the channel.

A livelock is considered to transition to a deadlock
if the channel becomes unreachable, in which case
the fibre will become Dead through Starvation (or Blockage),
disolving the deadlock. In other words, fibres cannot deadlock.


\section{Fibre Structure}
Each fibre consists of a single current continuation.
Each continuation may have an associated continuation
known as its caller. The initial continuation of a freshly
spawned fibre has no caller.

The closure of the caller relation leads to a linear
sequence of continuations starting with the current
continuation and ending with the initial continuation
of a freshly spawned fibre.

The main program consists of an initially Running
fibre with a specified initial continuation.

Continuations have the usual operations of a procedure.
They may return, call another procedure, spawn new
fibres, create channels, and read and write channels,
as well as the other usual operations of a procedure
in a general purpose programming language.

A continuation is reachable if it is the current
continuation of a reachable fibre, or the caller
of a reachable continuation.

\section{Continuation Structure}
\subsection{Continuation Data}
A continuation has associated with it the following
data:

\begin{description}
\item[caller] Another continuation of the same fibre which is
suspended until this continuation returns.
\item[data frame] Sometimes called the stack frame, contains
local variables the continuation may access.
\item[program counter] A location in the program code representing
the current point of this continuations execution or suspension
\end{description}

\subsection{Continuation operations}
The current continuation of a fibre executes a wide range of
operations including channel I/O, spawning new fibres,
calling a procedure, and returning.

\begin{description}
\item[call] Calling a procedure creates a new continuation
with its program counter set at the procedure entry point,
and a fresh data frame. The new continuation becomes the
current continuation, the current continuation suspends.
The new continuations caller field is set to the caller.
The current continuation program counter is set
to the pointer after the call instruction.

The effect is push an entry onto the fibres continuation chain.

\item[return] Returning from the current continuation causes
the owning fibres current continuation to be set to the
current continuations caller, if one exists, or the 
fibre to be marked Dead if there is no caller. Execution
of the suspended caller continues at its program counter.

The effect is to pop an entry off the fibre's continuation chain.

\item[read/write] Channel I/O suspends the current continuation
of a fibre until a matching operation from another fibre
synchronises with it. A read is matched by a write, and a write
is matched by a read.
\end{description}

By the rules of state change, channel I/O should be viewed
as performing a peer to peer neutral exchange of control:
the current fibre becomes suspended without losing its position
and hands control to another fibre. Later, control is handed
back and the fibre continues.

Coroutine based systems, therefore, operate by repeated exchanges
of control accompanied by data transfers in a direction independent
of the control flow, which sets coroutines aside from functions.

\section{Events}
Each state transfer of the fibration system may be considered
an event. However the key events are
\begin{itemize}
\item spawning
\item suicide
\item entry to a read operation
\item return from a read operation
\item entry to a write operation
\item return from a write operation
\end{itemize}

I/O synchronisation consists of suspension on entry
to a read or write operation, and simultaneously release
of suspension, or resumption, on matching write or read.

I/O suspension occurs when a fibre becomes Hungry or Blocked,
and resumption when it becomes Running or Active.

Fibrated systems are characterised by a simple rule: events are
totally ordered. The order may not be determinate. 

\section{Control Type}
The control type of a coroutine is defined as follows.
We assume the coroutine is spawned as a fibre, and each
and every read request is satisfied by a random value of
the legal input type. Write requests are also satisfied.
We cojoin entry and return from read into a single read
event, and entry and return from write into a single
write event, since we are only interested in the behaviour
of the fibre.

The sequence of all possible events which the fibre
may exhibit is the coroutines control type. Note, the
control type is a property of the coroutine (procedure).

\section{Encoding Control Types}
In general, the control type of a coroutine can be quite
complex. However for special cases, a simple encoding
can be given.

\subsection{One shots}
A one-shot coutine is one that exhibits a bounded number
of events before suiciding. The three most common one
shots are:

\begin{description}
\item[Value: type W] A coroutine which writes a single value to
a channel and then exits.
\item[Result: type R] A coroutine which reads a single value to
from channel and then exits.
\item[Function: type RW] A couroutine which reads one value
from a channel, calculates an answer, writes that
down a channel and then exits.
\end{description}

\subsection{Continuous devices}
A continuous coroutine is one which does not exit.
It can therefore terminate only by starvation or blockage.
The three most common kinds of such devices are
\begin{description}
\item[Source: type W+] Writes a continuous stream of values
to a channel.
\item[Sink: type R+] Reads a continuous stream of values
from a channel.
\item[Transducer] Reads and writes.
\end{description}

Because the sequence of events is a stream, we may use
convenient notations to describe control types.
If possible, a regular expression will be used.
Sometimes, a grammar will be required. In other cases
there is no simple notation for the behaviour of a coroutine.

We will use postfix \verb%+% for repetition.
 
\subsection{Transducer Types}
A transducer which read a value, write a value, then loops
back and repeats is called a {\em functional transducer},
it may be given the type (RW)+.

In a functional language, a partial function has no natural
encoding. There are two common solutions. The first is to
return an option type, say Some v, if there is a result,
or None if there is not. This solution involves modifying
the codomain. The other solution is to restrict the domain so
that the subroutine is a function.

Coroutines, however, represent partial functions naturally.
If a value is read for which there is no result, none is written!
The type of a {\em partial function transducer} is therefore
given by ((R+)W)+, in other words multiple reads may occur
for each write. Note that two writes may not occur in succession.

This type may also be applied to many other coroutines,
for example the list filter higher order function.

\subsection{Duality}
Coroutines are dual to functions. The core difference is that
they operate in time not space. Thus, in the dual space
a spatial product type becomes a temporal sequence.

Coroutines are ideal for processing streams. Whereas
function code cannot construct streams without laziness,
and cannot deconstruct them without eagerness, coroutines
are neither eager nor lazy.

One may view an eager functional application as driving a
value into a function to get a result, and a lazy application
as pulling a value into a function. Pushing value implies
eagerly evaluating it, pulling implies the value is calculated
on demand.

Coroutine simultaneously push and pull values across channels
and so eliminate the evaluation model dichotomy that plagues
functional programming. This coherence does not come for
free: it is replaced by indeterminate event ordering.

\section{Composition}
By far the biggest advantage of coroutine modelling is the 
ultimate flexibility of composition. Coroutines provide
far better modularity and reusability than functions,
but this comes at the price of complexity. You will observe
considerably more housekeeping is required to compose
coroutines than procedures or functions, because, simply,
there are more way to compose them.

A collection of coroutines can be regarded as black boxes
resembling chips on a circuit board, with the wires
connecting pins representing channels. So instead of using
variables and binding constructions, we can construct more
or less arbitrary networks.

\subsection{Pipelines}
The simplest kind of composition is the pipeline. It is a sequence
of transduces wired together with the output of one transducer
connected by a channel to the input of the next.

If the pipeline consists entirely of transducers is is an open
pipeline. If there is a source at one end and a sink at the other
it is a closed pipeline. Partially open pipelines can also exist.

The composition of two transducers has a type dependent on
the left and right transducer types. 

With a functional transducer, you would expect the composition
of (R1W1)+ with (R2W2)+ to be (R1W2)+ but this is not the case!

Consider, the left transducer performs R1, then W1, then right
performs R2. At this point it is not determinate whether left
or right proceeds. If left proceeds, we have R1 again, then W1.
then right proceeds and performs W2 before coming back to read R2,
and what happens next is again indeterminate. The sequence is
therefore R1, W1/R2, R1, W1/R2 which shows R1 can be read twice
before W2 is observed. We have written w/r here to indicate synchronised
events which are abstracted away when describing the observable
behaviour of the composite.

Clearly, (R1?R1W2?W2)+ contains the set of possible event sequences,
but then (R1+R2+)+ contains it, and therefore the set of possible
event sequences as well. So we should seek the most precise, or
{\em principal} type of the composite.

We can calculate the type from the operational semantics.
At any point in time, the system must be in one of a finite
number of states. Where we have indeterminacy, the transitions
out of a given state are not fully specified. The result is
clearly a non-deterministic finite state automaton.

We must observe, such an automaton corresponds to (one or more) 
larger deterministic finite state automata. This is an important
result because it has practical implications: it means we can
pick a DFA and use it to optimise away abstracted synchronisation
points. In other words, we build a fast model of the system
by inlining and using shared variables instead of channels,
and then eliminate the variables by functional composition.

This is the primary reason we insist on indeterminate behaviour:
it allows composition to be subject to a reduction calculus.

\section{Felix Implementation}
The following functions and procedures are provided in Felix:

\begin{minted}{felix}
spawn_fthread: 1 -> 0
mk_ioschannel_pair[T]: 1 -> ischannel[T] * oschannel[T];
read[T]: ischannel[T] -> T
write[T]: oschannel[T] * T -> 0
\end{minted}

In the abstract, channels are bidirectional and untyped.
However we will restrict our attention to channels
typed to support either read (ischannel) or write (oschannel)
of a value of a fixed data type.

The following shorthand types are available:

\begin{minted}{felix}
%<T    ischannel[T]
%>T    oschannel[T]
\end{minted}

More advanced typing exploiting channel capabilities are
discussed later.

Simple example program:

\begin{minted}{felix}
proc demo () {
  var inp, out = mk_ioschannel_pair[int]();

  proc source () {
    for i in 1..10 perform write (out,i);
  }

  proc sink () {
    while true do
      var j = read inp;
      println$ j;
    done
  }

  spawn_fthread source;
  spawn_fthread sink;
}
demo();
\end{minted}

In this program, we create a channel with an input and
output end typed to transfer an int. The source coroutine
writes the integers from 1 through to 10 inclusive to
the write end of the channel, the sink coroutine
reads integers from the channel and prints them.

The main fibre calls the demo procedure which launches
two fibres with initial continuations the closures of
the source and sink procedures.

When demo returns, the main fibre's current continuation
no longer knows the channel, so the channel is not reachable
from the main fibre. 

The source coroutine returns after sending 10 integers to
the sink via the channel.  When a fibre no longer has
a current continuation, returning to the non-existent
caller causes the fibre to no longer have a legal
state. This is known as suicide.

After the sink has read the 
last value, it becomes permanently Hungry. The sink 
procedure dies by starvation.

All fibres which die do so either by suicide, starvation,
or blockage. Dead fibres will be reaped by the garbage
collector provided they're unreachable. It is important
for the creator of fibres and their connecting channels
to forget the channels to ensure this occurs.

Unlike typical pre-emptive threading systems, deadlock
is not an error. However a lock up which should lead to
reaping of fibres but which fails to do so because
they remain reachable is univerally an error. This is
known as a livelock: it leads to zombie fibres.

This usually occurs because some other fibre is statically
capable of resolving the lockup, but does not do so 
dynamically. To prevent livelocks, variables holding
channel values to which no I/O will occur dynamically
should also go out of scope.
 
\end{document}
