\documentclass[oneside]{book}
\usepackage{color}
\definecolor{bg}{rgb}{0.95,0.95,0.95}
\definecolor{emphcolor}{rgb}{0.5,0.0,0.0}
\newcommand{\empha}{\bf\color{emphcolor}}
\usepackage{parskip}
\usepackage[newfloat=true]{minted}
\usepackage{amsmath}
\usepackage{amssymb}
\usepackage{amscd}
\usepackage{makeidx}
\usepackage{tikz}
\usetikzlibrary{shapes,shadows,arrows}
\makeindex
\usemintedstyle{friendly}
\setminted{bgcolor=bg,xleftmargin=20pt}
\usepackage{hyperref}
\hypersetup{pdftex,colorlinks=true,allcolors=blue}
\newcommand*{\fullref}[1]{\hyperref[{#1}]{\autoref*{#1} \nameref*{#1}}}
\usepackage{hypcap}
\usepackage{caption}
\DeclareMathOperator{\quot}{div}
\DeclareMathOperator{\rmd}{rmd}
\title{Programming with Coroutines}
\author{John Skaller}
\begin{document}
\maketitle
\tableofcontents
\chapter{Introduction}
Coroutines are not a new concept, however they have been ignored for
far too long. They solve many programming problems in a natural way and
any decent language today should provide a mix of coroutines and procedural
and functional subroutines, as well as explicit continuation passing.

Alas, since no such system exists to my knowledge I have had to create
one to experiment with: Felix will be used in this document simply
because there isn't anything else!

\section{What is a coroutine?}
A {\em coroutine} is basically a procedure which can be {\em spawned} to begin
a {\em fibre} of control which can be {\em suspended} and {\em resumed} under program
control at specific points. Coroutines communicate with each other
using {\em synchronous channels} to read and write data from and to other
coroutines. Read and write operations are synchronisation points,
which are points where a fibre may be suspended or resumed.

Although fibres look like threads, there is a vital distinction: multiple
fibres make up a single thread, and within that only one fibre is ever
executing. Fibration is a technique used to structure sequential programs,
there is no concurrency involved.

The most significant picture of the advantages of coroutines is thus: in a subroutine
based language there is a single machine stack. By machine stack, I mean that
there is an important {\em implicit} coupling of control flow and local variables.
In the abstract, a subroutine call passes a continuation of the caller to 
the callee which is saved along with local variables the callee allocates,
so that the local variables can be discarded when the final result is
calculated, and then passed to the continuation. This technique may be
called {\em structured programming}. With coroutines, the picture is simple:
each fibre of control has its own stack. Communication via channels exchanges data
and control between stacks.

Coroutines therefore leverage control and data coupling in a much more
powerful and flexible manner than mere functions, reducing the need for
state to be preserved on the heap, thereby making it easier to construct
and reason about programs.

For complex applications, the heap is always required.


\section{A Simple Example}
The best way to understand coroutines and fibration is to have a look
at a simple example. 

\subsection{The Producer}
First, we make a coroutine procedure which writes the integers
from 0 up to but excluding 10 down a channel.

\begin{minted}{felix}
proc producer (out: %>int) () {
  for i in 0..<10 
    perform write (out, i);
}
\end{minted}

Notice that as well as passing the output channel argument \verb%out%,
there is an extra unit argument \verb%()%. This procedure terminates
after it has written 10 integers. The type of variable \verb%out% is
denoted \verb$%>int$ which is actually short hand for \verb%ochannel[int]%,
which is an output channel on which values of type \verb%int% may
be written.

\subsection{The Transducer}
Next, we make a device which repeatedly reads an integer, squares it,
and writes the result. It is an infinite loop, this coroutine never
terminates of its own volition. This is typical of coroutines.

\begin{minted}{felix}
proc transducer (inp: %<int, out: %>int) () {
  while true do
    var x = read inp;
    var y = x * x;
    write (out, y);
  done
}
\end{minted}

Here, the type of variable \verb%inp% is
denoted \verb$%<int$ which is actually short hand for \verb%iochannel[int]%,
which is an input channel from which values of type \verb%int% may
be read.

\subsection{The Consumer}
Now we need a coroutine to print the results:

\begin{minted}{felix}
proc consumer (inp: %<int) () {
  while true do
    var y = read inp;
    println y;
  done
}
\end{minted}

Each of these components is a coroutine because it is a procedure
which may perform, directly or indirectly, I/O on one or more synchronous
channels.

\subsection{Purity}
The first two coroutines are {\em pure} because they depend only on their
arguments, and interact with the outside world entirely through 
synchronous channels. They do not modify variables in their environment,
and they do not depend on variables in their environment. The consumer,
however, has a side effect, namely printing values to standard output.

Purity is an important property which provides modularity and encapsulation and allows
one to reason locally. This is a vital information hiding property which is also
possessed by pure functions, where it is known as {\em functional abstraction}. 

They key idea of functional abstraction is that an approximation of the
function semantics is represented by the function type. For example
the functions \verb%modulus% and \verb%argument%

\begin{minted}{felix}
fun modulus (z:dcomplex) : double => sqrt (z.x^2 + z.y^2);
fun argument (z:dcomplex) : double => arctan2 (z.y, z.x);
\end{minted}

both have type \verb% dcomplex -> double%, so the type is only an approximation
which forgets some details of the function, which is the usual meaning
of abstract. Never-the-less the type is useful to allow the type checker
to prevent calling these functions on an integer, but more importantly
it allows for higher order functions:

\begin{minted}{felix}
fun map (x:list[dcomplex]) (f:dcomplex->double) =>
  match x with
  | Empty => Empty[double]
  | Cons (head, tail) => Cons (f head, map f tail)
  endmatch
;
\end{minted}

This function will take a list of \verb%dcomplex% and apply either
\verb%modulus% or \verb%argument% or any other function with
the type \verb%dcomplex->double% to the list to produce a list
of \verb%double% safely: the point is that this map function
does not need to know the full semantics of the argument to which
the parameter \verb%f% is bound, only that it has the correct type.

For coroutines,
we would call this cofunctional abstraction, however there's a problem: functions are
abstracted to function types. However the behaviour of a coroutines depend not just
on the data type of the channels, but also on the order in which operations
are performed on these channels, and that information should be approximated
and symbolised by a {\em control type}. Alas, 

\noindent {\bf we do not have a suitable type system.}

\subsection{Synchronous Channel Construction}
Now, let us see how we can use these coroutines in the obviously
intended way! First we have to make some channels to connect
the devices:

\begin{minted}{felix}
proc doit () {
  var ich1, och1 = mk_ioschannel_pair[int]();
  var ich2, och2 = mk_ioschannel_pair[int]();
\end{minted}

Note, we have only created two channels here! But we have
made two interfaces to the same channel, the first input,
and the second output.

\subsection{Connecting Devices with Channels}
Now we can connect the devices to the channels:

\begin{minted}{felix}
  var p = producer och1;
  var t = transducer (ich1, och2);
  var c = consumer (ich2);
\end{minted}

We have created procedure closures which bind the channel arguments
to the procedures so that now the three closures all have the type
\verb%1->0%, where 1 is also named as \verb%unit% and 0 is named as \verb%void%,
which is required for the next step.

\subsection{Spawning Fibres}
Now we spawn active fibres from the coroutine closures:

\begin{minted}{felix}
  spawn_fthread p;
  spawn_fthread t;
  spawn_fthread c;
}
doit();
\end{minted}

What we have done here is spawn three fibres which then communicate
via the connected channels. The configuration in a series is called
a {\em pipeline} and corresponds directly to functional composition.

\subsection{Termination}
Now you may wonder, how does it all end? What happens is that
when the producer terminates by a procedural return which is
called {\em suicide}. The transducer tries to read a value
which is never going to come. The transducer is said to {\em starve}.
The consumer also waits forever for a value from the transducer
which is never going to come, because the transducer is starving,
so the consumer also starves.

It is also possible for a coroutine to {\em block}. This happens when
it tries to write a value which will never be read. Lets modify
our example to see: an infinite production stream:

\begin{minted}{felix}
proc producer (out: %>int) () {
  var i = 0;
  while true do
    write (out, i);
    ++i;
  done
}
\end{minted}

but a limited sample of data are printed:

\begin{minted}{felix}
proc consumer (inp: %<int) () {
  for i in 0..<10 do
    var y = read inp;
    println y;
  done
}
\end{minted}

Now, the transducer blocks when the consumer terminates, and thus
the producer blocks because the transducer has.

The astute programmer will have a number of questions!
When a pre-emptive thread starves or blocks, it is a serious
problem. Have we made a mistake with our fibres?

Here, you start on your journey to a major paradigm shift!
Blockage and starvation are not an error with coroutines,
its normal, expected, and desirable! This is, in fact,
the main way that we organise termination!

Before I can explain this, however, I have to back step a bit!

\section{Garbage Collection and Reachability}
Felix runs a garbage collector similar to most functional programming
languages. What a collector does is maintain a specified set of root
objects, and finds all the objects to which there is a pointer
in one of the roots. It then expands the set to include all the objects
for which there is a pointer in one of those objects, and so on.
If an object A has a pointer to an object B, we say B is directly
reachable from A. If B then has a pointer to C, then C is said to be
reachable from A, by first visiting B. The complete set of objects
reachable from the designated roots is the transitive closure of the
reachability relation. The other objects which are not reachable
are garbage and are deleted. There's no way to refer to such an object,
since there are no pointers to it in the roots, or any object reachable
from the root.

\begin{figure}[h]
\begin{center}
\tikzstyle{cell} = [rectangle,minimum width=2cm,minimum height=1cm,text centered,draw=black]
\tikzstyle{arrow} = [thick,->,>-stealth]
\begin{tikzpicture}[node distance=3cm]
\node(A)[cell]{A};
\node(B)[cell,right of=A]{B};
\node(C)[cell,right of=B]{C};
\draw[arrow](A) -- (B);
\draw[arrow](B) -- (C);
\end{tikzpicture}
\caption{Reachability}
\end{center}
\end{figure}

Now, the secret of Felix coroutines is as follows: when you spawn
a coroutine, the resulting fibre is reachable by the system,
but it is {\em not} reachable from the caller. There is no "thread-id"
returned when a coroutine is spawned, if you want to communicate
with it you have to use a channel. The coroutine is named, the 
fibre spawned, however, is {\em anonymous}.

Now what happens is very simple but you will have to concentrate
to get it! Coroutines passed channels can reach the channel.
Any procedure which stores the channel can reach the channel.
But the channel is an object and initially it can't reach anything.
However when a coroutine performs I/O on the channel it can be
suspended. If a read is done, and there is not yet a matching write,
the fibre is suspended by adding it to the channel. Now the channel
can reach the fibre. At the same time the system {\em forgets} the
fibre. The system keeps a list of active fibres, but a suspended
fibre is not active so it is forgotten.

A read operation is matched by a write, and a write operation
is matched by a read. When a matching I/O operation is performed
on a channel it means that the other operation that matches it
has already been performed by another fibre. In this case,
the channel forgets that fibre, and {\em both} that fibre and the
one performing the matching operation become active and reachable
by the system.

A more detailed explanation follows. A formal definition of the
precise execution semantics is given in \fullref{Coroutine Semantics}.

\section{Execution Model}
When Felix starts your program, the machine stack is reachable,
and so any object with a pointer on the machine stack is reachable.
In addition, your initial mainline code is implicitly a coroutine,
which is spawned automatically creating a fibre object
which contains a pointer to the fibre's initial continuation.
The fibre is running, so it is also reachable.

All your top level variables are stored in an object called
the {\em thread frame}. Continuation objects contain a pointer
to the thread frame so that the procedure can access the global
variables. A continuation object is also known as the procedure
{\em activation record} or {\em data frame}, or, historically,
its {\em stack frame}. As well as a pointer to the thread
frame, a continuation object also contains a pointer to the
most recent activation record of its ancestors, in fact the
thread frame may be consider a universal ancestor.

Vitally, continuation objects also contain a value known as
the program counter. This value is the current location at which
the continuation is executing, it always points into the code
of the procedure the frame represents. When a subroutine is
called, the program counter is set to the statement after the
subroutine call, a new continuation is created for the subroutine,
its program counter is set to the first statement, and a back pointer
to the caller continuation is stored. The back pointer, together with
the program counter of the caller, are together known as the subroutine
{\em return address}. It represents the continuation which the
subroutine resumes when the subroutine itself is complete.

Then the current continuation of the fibre is 
changed to the new continuation.

\begin{figure}
\begin{minted}{C++}
#if FLX_CGOTO
  #define FLX_LOCAL_LABEL_VARIABLE_TYPE void*
  #define FLX_PC_DECL void *pc;
#else
  #define FLX_PC_DECL int pc;
  #define FLX_LOCAL_LABEL_VARIABLE_TYPE int
#endif

\end{minted}
\caption{Code Address}
\end{figure}

\begin{figure}
\begin{minted}{C++}
struct con_t 
{
  FLX_PC_DECL               // interior program counter
  struct _uctor_ *p_svc;    // service request

  virtual con_t *resume()=0;// computation step
  con_t * _caller;          // return address
};
\end{minted}
\caption{Continuation base}
\end{figure}

\begin{figure}
\begin{minted}{C++}

struct fthread_t 
{
  con_t *cc;                // current continuation
};
\end{minted}
\caption{Fibre}
\end{figure}

\begin{figure}
\begin{minted}{C++}
struct slist_node_t {
  slist_node_t *next;
  fthread_t *data;
};

struct slist_t {
  gc::generic::gc_profile_t *gcp; // garbage collector
  struct slist_node_t *head;
};

struct schannel_t
{
  slist_t *waiting_to_read;  // fthreads waiting for a writer
  slist_t *waiting_to_write; // fthreads waiting for a reader
};
\end{minted}
\caption{Synchronous Channel}
\end{figure}

When a return statement is executed, the backpointer to the
caller is stored into the fibre object, and execution continues
with the caller at the statement after the subroutine call.

Thus, the continuations form a singly linked list which operates
like a stack. The continuation objects are heap allocated and
the data structure is known as a {\em spaghetti stack}.
In principle, a continuation also has a pointer to the most recent
activation record of its parent, which has a pointer to its 
parent, until the list terminates with the thread frame,
so there are {\bf two} interleaved lists here: one representing
the call chain, and one representing the static nesting structure:
that is what make it a spaghetti stack. In Felix, pointers to all
the ancestors are stored in the continuation object to improve
access time to ancestoral variables, at the cost of passing them
all to each child (however the optimiser does lots of magic).

Each of the frame pointers mentioned is known to the garbage collector
and so a single reachable running procedure defines a transitive
closure of reachable objects. Note that in addition, any variable
containing a Felix pointer obtained from a manual heap allocation
ensures the heap object is reachable if the variable is in a 
reachable frame. In addition, any reachable pointer which points
anywhere into an object ensures the object is reachable.
If the pointer is not to the first byte, it is called
an {\em interior pointer}. Note that in Felix a pointer
"one past the end" of an object does not make the object
reachable!

Now all this explains, technically, something easy to state
loosely: if you can access an object it is reachable.
In addition if the {\em system} can access the object it
is reachable.

Now what I have described so far does not explain fibres.
The currently running fibre, and all those deemed active
are reachable by the system. When the currently running fibre
performs an unmatched synchronous channel I/O operation,
either a read or a write, it is added to the channel's
list of suspended fibres and is removed from the set of
fibre the system can reach directly. So the fibre can now
only be reached from the channel. So it will be garbage
unless another active fibre can reach the channel.
After all since the I/O operation is unmatched, if another
fibre can't see the channel, there is no fibre that
can satisfy the I/O request.

When a fibre is suspeneded by a read or write operation,
the program counter of its current continuation is set
to the statement after the I/O operation. If the operation
is later matched, the address of the data being transmitted
is transfered from the writer to the reader, and the two
fibres both made active so that they will continue
at the statement after the I/O operation.

What is vital to realise now is that each fibre has its
own spaghetti stack. So when control is exchanged
from one fibre to another:

\noindent {\bf control exchange is effected by stack swapping}

Of course, we mean the heap allocated spaghetti stacks.
You can swap machine stacks too: this is done by the host
operating system scheduler and the entities being context
switched are known as threads. The swaps are premptive,
and several stacks can be running at once if you have a 
multi-core CPU. Pre-emptive threads are much harder to use
than fibres, and the context switches are much more
expensive. They are greatly overused in many programs
for purpose of obtaining control inversion because
the host language is deficient and does not support
coroutines. This deficiency is shared by almost all
production and research programming systems!

\begin{figure}
\tikzstyle{cell} = [rectangle,minimum width=2cm,minimum height=1cm,text centered,draw=black]
\tikzstyle{blank} = [rectangle,minimum width=2cm,minimum height=1cm,text centered]
\tikzstyle{arrow} = [thick,->,>-stealth]
\begin{tikzpicture}[node distance=3cm]
\node(sys)[cell]{System};
\node(sys1)[blank,below of=sys]{};
\node(sys2)[blank,below of=sys1]{};
\node(sys3)[blank,below of=sys2]{};
\node(sys4)[blank,below of=sys3]{};
\node(prog)[cell,right of=sys1]{Fibre 0};
\node(doit)[cell,right of=prog]{doit};
\node(main)[cell,right of=doit]{Mainline};
\node(prod)[cell,right of=sys2]{Fibre 1};
\node(prod1)[cell,right of=prod]{producer};
\node(sys3a)[blank,right of=sys3]{};
\node(sys3b)[blank,right of=sys3a]{};
\node(chan)[cell,right of=sys3b]{schannel};
\node(cons)[cell,right of=sys4]{Fibre 2};
\node(cons1)[cell,right of=cons]{consumer};
\draw[arrow](sys) -- (prog);
\draw[arrow](sys) -- (prod);
\draw[arrow](sys) -- (cons);
\draw[arrow](prog) -- (doit);
\draw[arrow](doit) -- (main);
\draw[arrow](doit) -- (chan);
\draw[arrow](doit) -- (chan);
\draw[arrow](prod) -- (prod1);
\draw[arrow](cons) -- (cons1);
\draw[arrow](prod1) -- (chan);
\draw[arrow](cons1) -- (chan);
\end{tikzpicture}
\caption{Reachability: After Spawning}
\end{figure}


\begin{figure}
\tikzstyle{ucell} = [rectangle,minimum width=2cm,minimum height=1cm,text centered,draw=black!20]
\tikzstyle{cell} = [rectangle,minimum width=2cm,minimum height=1cm,text centered,draw=black]
\tikzstyle{blank} = [rectangle,minimum width=2cm,minimum height=1cm,text centered]
\tikzstyle{arrow} = [thick,->,>-stealth]
\tikzstyle{uarrow} = [thick,->,>-stealth,draw=black!20]
\begin{tikzpicture}[node distance=3cm]
\node(sys)[cell]{System};
\node(sys1)[blank,below of=sys]{};
\node(sys2)[blank,below of=sys1]{};
\node(sys3)[blank,below of=sys2]{};
\node(sys4)[blank,below of=sys3]{};
\node(prog)[ucell,right of=sys1]{Fibre 0};
\node(doit)[ucell,right of=prog]{doit};
\node(main)[ucell,right of=doit]{Mainline};
\node(prod)[cell,right of=sys2]{Fibre 1};
\node(prod1)[cell,right of=prod]{producer};
\node(sys3a)[blank,right of=sys3]{};
\node(sys3b)[blank,right of=sys3a]{};
\node(chan)[cell,right of=sys3b]{schannel};
\node(cons)[cell,right of=sys4]{Fibre 2};
\node(cons1)[cell,right of=cons]{consumer};
\draw[arrow](chan) -- (prod);
\draw[arrow](sys) -- (cons);
\draw[uarrow](doit) -- (chan);
\draw[arrow](prod) -- (prod1);
\draw[arrow](cons) -- (cons1);
\draw[arrow](prod1) -- (chan);
\draw[arrow](cons1) -- (chan);
\end{tikzpicture}
\caption{Reachability: Mainline completed, after Write, before Read}
\end{figure}


\begin{figure}
\tikzstyle{ucell} = [rectangle,minimum width=2cm,minimum height=1cm,text centered,draw=black!20]
\tikzstyle{cell} = [rectangle,minimum width=2cm,minimum height=1cm,text centered,draw=black]
\tikzstyle{blank} = [rectangle,minimum width=2cm,minimum height=1cm,text centered]
\tikzstyle{arrow} = [thick,->,>-stealth]
\tikzstyle{uarrow} = [thick,->,>-stealth,draw=black!20]
\begin{tikzpicture}[node distance=3cm]
\node(sys)[cell]{System};
\node(sys1)[blank,below of=sys]{};
\node(sys2)[blank,below of=sys1]{};
\node(sys3)[blank,below of=sys2]{};
\node(sys4)[blank,below of=sys3]{};
\node(prog)[ucell,right of=sys1]{Fibre 0};
\node(doit)[ucell,right of=prog]{doit};
\node(main)[ucell,right of=doit]{Mainline};
\node(prod)[ucell,right of=sys2]{Fibre 1};
\node(prod1)[ucell,right of=prod]{producer};
\node(sys3a)[blank,right of=sys3]{};
\node(sys3b)[blank,right of=sys3a]{};
\node(chan)[ucell,right of=sys3b]{schannel};
\node(cons)[ucell,right of=sys4]{Fibre 2};
\node(cons1)[ucell,right of=cons]{consumer};
\draw[uarrow](chan) -- (cons);
\draw[uarrow](doit) -- (chan);
\draw[uarrow](cons) -- (cons1);
\draw[uarrow](prod1) -- (chan);
\draw[uarrow](cons1) -- (chan);
\end{tikzpicture}
\caption{Reachability: Producer completed, Consumer starved, Program finished}
\end{figure}




\section{Indeterminacy}
When fibres synchronise with matching I/O operations, both become
active but only one actually starts executing. Which one is 
{\em indeterminate}. Felix always runs the reader first, but
in the abstract semantics you are not allowed to know that.
Indeterminacy is as close to concurrency as we can get with
a sequential program and its vital not only for optimisation,
but to ensure the programmer does not get bogged down depending
on implementation details.

So now that you understand reachability, you will begin to 
understand what happens when a fibre starves. Provided there
is no active fibre which can reach the channel, then since
the only object which can reach the fibre is the channel,
which is unreachable, the starving fibre is also unreachable.
So it is garbage collected!

Note {\em very carefully} that it is {\em absolutely essential}
that channels only be reachable by fibres that wil use them.
Go back and look carefully at the \verb%doit% procedure:

\begin{minted}{felix}
proc doit () {
  var ich1, och1 = mk_ioschannel_pair[int]();
  var ich2, och2 = mk_ioschannel_pair[int]();
  var p = producer och1;
  var t = transducer (ich1, och2);
  var c = consumer (ich2);
  spawn_fthread p;
  spawn_fthread t;
  spawn_fthread c;
}
doit();
\end{minted}

The four channel end points are known to this procedure, so whilst
this procedure is active, those channels are reachable. Indeed
the three closures \verb%p,t,c% are bound to these channels,
and the procedure knows them too. So the fibres spawned by this
procedure may be reachable whilst the procedure itself is active.

Now, when you spawn a fibre, what happens? Does the spawned fibre
run immediately, or does the spawning procedure continue?

Did you guess? In the abstract semantics, it is indeterminate!
You're not allowed to design code that depends on which one runs
first. In Felix, the spawned procedure runs first, but that's an
implementation detail!

So what happens here is that sometime or other, the procedure
will return, and the channels it could reach will no longer be
reachable because the procedure's local data frame is no longer reachable.

And then, the procedure's data frame will be reaped by the collector,
and, when the spawned fibres finally terminate, starve or block, they will
also be reaped.

If you're getting the picture you may well wonder how the program
as a whole terminates, and the answer is: in Felix the mainline
is a coroutine! It is not a subroutine. In fact in Felix,
all subroutines are, in the abstract, coroutines. The normal
procedural subroutines are just coroutines that do not do channel I/O.

\chapter{Coroutine Basics}

\section{Syntax Extensions}
Felix has two syntax extensions designed to so coroutines are easier to use.

\subsection{The {\tt chip} definition}
This extensions encourages a picture of coroutines as integrated
circuits, even though that is not really accurate.

\phantomsection
\label{fig:prodtranscons1}
\begin{minted}{felix}
chip producer 
  connector io
    pin out: %>int
{
  for i in 0..<10 
    perform write (io.out, i);
}
\end{minted}
\begin{minted}{felix}
chip transducer 
  connector io
    pin inp: %<int
    pin out: %>int 
{
  while true do
    var x = read io.inp;
    var y = x * x;
    write (io.out, y);
  done
}
\end{minted}
\begin{minted}{felix}
chip consumer 
  connector io
    pin inp: %<int
{
  while true do
    var y = read io.inp;
    println y;
  done
}
\end{minted}

Here \verb%connector% names an argument to the procedure
of record type. The fields of the record and specified with
the \verb%pin% clause. You can have more than one connector
phrase, each specifies a separate argument. Each \verb%chip%
has an aditional unit argument added automatically. The signatures
of the three chips above are:

\begin{minted}{felix}
  producer: (out: %out) -> 1 -> 0
  transducer: (inp: %<int, out: %out) -> 1 -> 0
  consumer: (inp: %<int) -> 1 -> 0
\end{minted}

Note that this syntax uses a record type for the connector,
whereas our original coroutines used a plain type for one
parameter and a tuple for the two required for the transducer.

\subsection{The {\tt device} statement}
You can write:

\begin{minted}{felix}
device x = y;
\end{minted}

to construct a procedure closure of type \verb%unit->void%.
Actually, device is just a synonym for \verb%var%, and is
provided to make your look more like an electrical engineer
than a software engineer.


\subsection{The {\tt circuit} statement}
\subsubsection{The {\tt connect} clause}
A \verb%circuit% statement can be used to connect devices
and pins. It is an executable statement!

\begin{figure}[h]
\begin{minted}{felix}
circuit
  connect producer.out, transducer.inp
  connect transducer.out, consumer.inp
endcircuit
\end{minted}
\caption{Simple circuit connection}
\label{fig:circ1}
\end{figure}

This makes a pipeline from the chips. The connecting channels
are automatically created, as are the procedure closures required
to make devices. The resulting devices are then spawned.

You can list any number of comma separated device/pin pairs
in a connect clause. Felix finds the transitive closure of
connections and makes a channel to connect all those pins together.
The data type of all connected pins must be the same. If all are inputs
or all are outputs, the compiler will issue a warning (but it is not
an error!).

\subsubsection{The {\tt wire} clause}
There is also another clause you can use in a circuit statement:

\begin{minted}{felix}
circuit
  wire ch to dev.pin
endcircuit
\end{minted}

The \verb%wire% clause allows you to connect a known channel to a device.

\chapter{Core Components}
Every system needs a library! 
\section{Base Components}
\subsection{Blockers}
Here is a device to use when you have to connected
a writer to a channel, but want it to be unconnected.

\begin{minted}{felix}
chip writeblock[T]
  connector io
    pin inp : %<T
{
}
\end{minted}

And the corresponding reader:

\begin{minted}{felix}
chip readblock[T]
  connector io
    pin out: %>T
{
}
\end{minted}

These coroutines suicide immediately, so a writer is blocked,
or a reader starved, respectively.

\subsection{Universals}
This chip reads input forever but ignores it.

\begin{minted}{felix}
chip sink[T]
  connector io
    pin inp : %<T
{
  while true do
    var x = read (io.inp);
    C_hack::ignore (x);
  done
}
\end{minted}

Whereas this chip writes a fixed value forever.
It is the analogue of a value or constant function:

\begin{minted}{felix}
chip source[T] (a:T)
  connector io
    pin out: %>T
{
  while true perform write (io.out, a);
}
\end{minted}

\subsection{Adaptors}
Two key adaptors provide {\em lifts}:


\begin{minted}{felix}
chip source_from_list[T] (a:list[T])
  connector io
    pin out: %>T
{
  for y in a perform write (io.out,y);
}

chip bound_source_from_list[T] (a:list[T])
  connector io
    pin out: %>opt[T]
{
  for y in a perform write (io.out,Some y);
  while true perform write None[T];
}
\end{minted}

A lift is a way to go from the imperative/function model
of the world into the coroutine/stream model. These
two chips lift a list into a stream.

It is absolutely vital to understand how these two
lifts differ. The first lift is a pure lift
which simply starves any read trying to go past the
end of the list. There is no terminal value to tell the
reader the list has ended. Notice a finite number
of values is written by the first device, equal to the
number in the list. This is a {\em finite stream}.

The second device generates an infinite stream
by embedding a finite list in its head using 
an option type. The tail of the stream is an infinite
list of None's. The None values are terminators which
act to bound the list.

I will warn now, to understand how to use the first device
requires a paradigm shift. Having things drop dead without
any warning seems difficult to manage if you're used
to dealing with inductive data types in a functional
setting. We will see later, however, that it is natural.

Next, we have a basic adaptor for a function.

\begin{minted}{felix}
chip function[D,C] (f:D->C)
  connector io
    pin inp: %<D
    pin out: %>C
{
  while true do
    var x = read io.inp;
    var y = f x; 
    write (io.out, y);
  done
}
\end{minted}


This device is an example of the generic category 
of a {\em transducer}. 

\subsection{Drops}
A drop is a way to get out of the coroutine/stream model
back to your imperative functional model:

\begin{minted}{felix}
chip procedure[D] (p:D->0)
  connector io
    pin inp: %<D
{
  while true do 
    var x = read io.inp;
    p x;
  done
}
\end{minted}

The device is a sink which will typically
create side effects for each value read.

A special case is a list drop:

\begin{minted}{felix}
chip sink_to_list[T] (p: &list[T])
  connector io
    pin inp : %<T
{
  while true do
    var x = read (io.inp);
    p <- Cons (x,*p);
  done
}
\end{minted}

Now, we need an example to show how to use this drop!

\begin{minted}{felix}
include "std/control/chips";
open BaseChips;

var output = Empty[int];

device s = source_from_list ([1,2,3,4]);
device tr = function (fun (x:int)=>x*x);
device d = sink_to_list &output;
run  { 
  circuit
    connect s.out, tr.inp
    connect tr.out, d.inp
  endcircuit
};
println$ output;
\end{minted}
 
The critical thing to note is the \verb%run% procedure.
It spawns its argument procedure as the initial fibre
of a new fibre scheduler, and then waits until that scheduler
terminates due to a lack of active fibres.

So {\em within} the fibre system, we cannot detect the end
of the list, but from outside, we can detect it indirectly
by the fact that our circuit is no longer active.

The \verb%run% procedure first lifts out of the current
procedural/imperative mode into fibrated stream mode,
waits until it completes, and then drops back to procedural
mode. In other words it interfaces the two modes.

\verb%run% can be used in a procedure, in a coroutine,
or in a function. Run, in effect,  creates and pushes a scheduler
on a scheduler stack, waits until it completes, and then pops
back to the current scheduler.

\subsection{Avoiding lockup}
To avoid some cases of lockup we provide the buffer device:

\begin{minted}{felix}
chip buffer [T]
  connector io
    pin inp: %<T
    pin out: %>T
{
  while true do
    var x = read io.inp;
    write (io.out, x);
  done
}
\end{minted}

You can see this is a just a copy operation and is a special
case of the \verb%function% chip, which uses the identity
function. However in a fibrated setting, \verb%buffer% is not
semantically a no operation.

Here's an example:

\begin{minted}{felix}
include "std/control/chips";
open BaseChips;

chip out2
  connector io
    pin oa: %>int
    pin ob: %>int
{
  write (io.oa, 11);
  write (io.ob, 42);
}

chip in2 
  connector io
    pin ia: %<int
    pin ib: %<int
{
   var a = read a;
   var b = read b;
   println $ a - b;
}

// WOOPS, lock up!
//circuit
//   connect out2.ob, in2.ia
//   connect out2.oa, in2.ib
//endcircuit

device abuf = buffer;
device bbuf = buffer;
circuit
   connect out2.ob, bufb.inp
   connect out2.oa, bufa.inp
   connect in2.a, bufa.out
   connect in2.b, bufb.out
endcircuit
\end{minted}

This is a classic deadlock for threads. The writer writes $a$
first then $b$, then reader reads $b$ first, then $a$.
Adding the buffers removes the ordering dependency.
I added two buffers, although in this case only one is
required. Can you figure out which two pins have to connected
via a buffer?

Here is another useful chip, it copies a single value
from input to output then suicides. It is generally useful
to insert into pipelines that would otherwise be continuous
when you only want a one shot operation.

\begin{minted}{felix}
chip oneshot [T]
  connector io
    pin inp: %<T
    pin out: %>T
{
  write (io.out, read io.inp);
}
\end{minted}

Finally here are some convenience types:

\begin{minted}{felix}
typedef iopair_t[D,C] = (inp: %<D, out: %>C);

// source
typedef ochip_t[T] = (out: %>T) -> 1 -> 0;

// transducer
typedef iochip_t[D,C] = iopair_t[D,C] -> 1 -> 0;

// sink
typedef ichip_t[T] = (inp: %<T) -> 1 -> 0;

\end{minted}

which specify the type of three commonly used chips.

\chapter{Pipelines}
One of the most basic control structures you can build with
coroutines is the {\em pipeline}. This is a series connection
of transducers, the output of the left one of a pair connected
to the input of the right one. A pipeline of transducers is said
to be an {\em open pipeline}.

If a source is connected to the left end, and a sink to the right
end, it is a closed pipeline.

An open pipeline is semantically equivalent to a transucer with
additional buffering. A pipline closed on the left is a source,
and a pipeline closed on the right is a sink.
The syntax |-> is parsed to pipe (a,b).
We add overloads for chips with pins
named io.inp, io.out.

Here are the binary combinators:

This chip connects two transducers to form a new
transducer. Note, since we use the \verb%circuit%
statement, the pair of component coroutines are
actually spawned as fibres.

\begin{minted}{felix}
chip pipe[T,U,V] (a:iochip_t[T,U],b:iochip_t[U,V])
 connector io
   pin inp: %<T
   pin out: %>V
{
  circuit
    connect a.out,b.inp
    wire io.inp to a.inp
    wire io.out to b.out
  endcircuit
}
\end{minted}

Here we connect a source to a transducer to make
a new source:

\begin{minted}{felix}
chip pipe[T,U] (a:ochip_t[T],b:iochip_t[T,U])
 connector io
   pin out: %>U
{
  circuit
    connect a.out,b.inp
    wire io.out to b.out
  endcircuit
}
\end{minted}

Here, a transducer is connected to a sink
to form a new sink.

\begin{minted}{felix}
chip pipe[T,U] (a:iochip_t[T,U],b:ichip_t[U])
 connector io
   pin inp: %<T
{
  circuit
    connect a.out,b.inp
    wire io.inp to a.inp
  endcircuit
}
\end{minted}

Finally, connecting a source to a sink results in
a closed pipeline. Closed pipelines are equivalent in
some sense to subroutines in that they can only be observed
for their side effects.

\begin{minted}{felix}
// source to sink
proc pipe[T] (a:ochip_t[T],b:ichip_t[T]) ()
{
  circuit
    connect a.out,b.inp
  endcircuit
}
\end{minted}

Note carefully, this last operator is a procedure not a chip!

Felix provides the infix symbol \verb%|->% for the \verb%pipe% operator.
An example of use, we can say:

\begin{minted}{felix}
producer |-> transducer |-> consumer;
\end{minted}

given the chips of \ref{fig:prodtranscons1}
instead of the circuit statemnent \ref{fig:circ1}.

\subsection{The lift functor}
Pipelining is associative up to buffering. The exact structures 
spawned may differ and the order of execution may differ, but
the ordering always conforms to the abstract semantics.

There is a mapping between function compositions and pipelines,
and this mapping is structure preserving. Given a sequence of
functions of types suitable for composition
$$f_1, f_2, f_3 \dots$$
then writing $\Phi$ for the \verb%function% chip, we have
$$\Phi(f_1 \odot f_2 \odot f_3 \dots) \cong
\Phi f_1 \mapsto \Phi f_2 \mapsto \Phi f_3 \dots)
$$
where $\odot$ is reverse function composition.

In other words, it is structure preserving, and thus a categorical {\em functor}.
It is called the {\em lift} functor because it lifts functional
code into cofunctional code, that is, functional stuff is lifted
into semantically equivalent coroutine based code. You can also drop
any pipeline to a function composition, so the two systems
are isomorphic.

The key point, which we are yet to demonstrate, is that
pipelines are not the only kind of circuits you can make.
Cofunctional programming {\em subsumes} functional programming.
Its more flexible and more powerful.

\subsection{Linear Flow}
Linear flow circuits are an extension of the pipeline
concept which allows data to flow from sources to sinks
in an acyclic network. Lets look at an example with two
sources:

\begin{minted}{felix}
device A = source_from_list ([1,2,3]);
device B = source_from_list ([5,6,7]);
chip add
  connector io
    a: %<int
    b: %>int
    sum: %>int
{
  while true do
    var a = read io.a;
    var b = read io.b;
    write (io.sum, a + b);
  done
}
circuit
  connect A.out, add.a
  connect B.out, add.b
  connect add.sum, consumer.inp
endciruit
\end{minted}

where we have used our original consumer to print the results.
There is an important thing to observe here: the order in which
our \verb%add% chip reads it input does not matter {\em in
this case} because it is connected to two {\em independent} sources.

You can probably see that given any binary operators represented
as chips, we can construct a calculate with a tree like structure
to perform the calculation.

\begin{minted}{felix}
chip sub
  connector io
    a: %<int
    b: %>int
    diff: %>int
{
  while true do
    var a = read io.a;
    var b = read io.b;
    write (io.diff, a - b);
  done
}
\end{minted}


If you are familiar with functional programming concepts, you may
ask whether these functions are eagerly or lazily evaluated.
Eager evaluation means the arguments are evaluated first,
before the function is called, so if such an evaluation fails
to terminate, the function call never happens, and we can say that
the whole application fails to terminate.

With lazy evaluation, the arguments are only evaluated when they're
actually needed. So if an argument which would be nonterminating
is not actually needed, the function application can still succeed.

Because coroutines provide explicit control flow, the evaluation
strategy depends on the way you write the coroutine. To put this
another way, there is no built in preference for either eager or
lazy evaluation. We will demonstrate by showing an important
operator written two different ways. First the eager variant:

\begin{minted}{felix}
chip eagerconditional
  connector io
    pin condition: %<bool
    pin trueval: %<int
    pin falseval: %<int
    pin result: %>int
{
  var c = read io.condition;
  var t = read io.trueval;
  var f = read io.falseval;
  write (io.result, if c then t else f);
}
\end{minted}

and now the lazy variant:

\begin{minted}{felix}
chip lazyconditional
  connector io
    pin condition: %<bool
    pin trueval: %<int
    pin falseval: %<int
    pin result: %>int
{
  var c = read io.condition;
  if c do
    var t = read io.trueval;
    write (io.result, t);
  else 
    var f = read io.falseval;
    write (io.result, t);
  done
}
\end{minted}

If the eager chip starves on either the read of the true value
or the false value, then any reader of the result also starves,
no matter what is read for the condition. However the lazy
chip only ever reads the value it is required to output,
and so only starves if the the read on that channel starves.
If it doesn't not, then it doesn't matter if a read on the
other channel starves, because we never actually read it.

Another interesting chip is this one:

\begin{minted}{felix}
chip choose
  connector io
    pin condition: %<bool
    pin value: <%int
    pin truecont: %>int
    pin falsecont: %>int
{
  var c = read io.condition;
  var v = io.value;
  if c do
    write (io.truecont, v);
  else
    write (io.falsecont, v);
  done
}
\end{minted}

This is a very important chip to understand! What it does is read
a condition and a value and write that value down one of two 
channels, depending on the condition.

At the other end of the two outputs
there may well be two different chips reading the result,
one to handle each of the two conditions. So this chip is like
a conditional goto chip, or a switch, in that it choses how
the rest of the program will proceed by selecting a data
path. Whichever path is chosen, the continuation suspended
at the end of the channel will continue execution. So passing
an output channel to a chip is an abstract way of passing
a continuation.

I say abstract because the actual chip which resumes control
on reading a value from the channel is dependent entirely
on how the circuit is connected. It doesn't depend on the
actual channel passed directly, but what is connected to
the other end.

Critically, the \verb%choose% chip enforces lazy evaluation
because only one of the channels is written to, what's
connected to the other end will only be activated if its 
input channel is selected for the write. In particular
I want you, the reader, to see that channels are not
merely ways to send data around, rather, they're ways to
transmit {\em control}. In particular, networks of connected
chips have a shape called a {\em control structure}.

So now we have looked at extensions to the pipeline concept
in which we have chips with multiple inputs and outputs,
and we are going to demonstrate how to handle the partial
function division:

\begin{minted}{felix}
chip divide
  connector io
    pin numerator: %<int
    pin denominator: %<int
    pin quotient: %>int
    pin divisionbyzero: %>int
{
  var n = read io.numerator;
  var d = read io.denominator;
  if d == 0 do
     write (io.divisionbyzero, numerator);
  else
     write (io.quotient, numerator/denominator);
  done
}
\end{minted}

This is an important chip because it shows how to handle
a partial function correctly, by providing an error channel.

Imagine we have to compute the formula:
$${x + y\over x - y}+1$$
We can use this:

\begin{minted}{felix}
var x = 1;
var y = 1;
device xc = x.source;
device yc = y.source;
device one = 1.source;

device add1 = add;
device add2 = add;

chip error
  connector io
    pin inp: %>int
{
  var x = read io.inp;
  println "Division of " + x.str + " by zero";
}

circuit
  connect add1.a, xc.out
  connect add1.b, yc.out
  connect sub.a, xc.out
  connect sub.b, yc.out
  connect div.numerator, add1.sum
  connect div.denominator, sub.diff
  connect div.quotient, add2.a
  connect one.out, add2.b
  connect add.sum, consumer.inp 
  connect div.divisionbyzero, error.inp
endciruit
\end{minted}

\begin{figure}[h]
\begin{center}
\tikzstyle{cell} = [rectangle,minimum width=0.7cm,minimum height=0.5cm,text centered,draw=black]
\tikzstyle{arrow} = [thick,->,>-stealth]
\begin{tikzpicture}[node distance=3cm]
\node(O)[cell]{1};
\node(X)[cell,below of=O]{x};
\node(Y)[cell, below of=X]{y};
\node(A1)[cell,right of=X]{add};
\node(S)[cell,right of=Y]{sub};
\node(D)[cell,right of=S]{div};
\node(A2)[cell,right of=D]{add};
\node(C)[cell,right of=A2]{consumer};
\node(E)[cell,below of=D]{error};
%\node(B)[cell,right of=A]{B};
%\node(C)[cell,right of=B]{C};
\draw[arrow](X) -- (A1);
\draw[arrow](Y) -- (A1);
\draw[arrow](A1) -- (D);
\draw[arrow](X) -- (S);
\draw[arrow](Y) -- (S);
\draw[arrow](S) -- (D);
\draw[arrow](D) -- (A2);
\draw[arrow](D) -- (E);
\draw[arrow](O) -- (A2);
\draw[arrow](A2) -- (C);
\end{tikzpicture}
\caption{Flow in simple formula}
\label{fig:flowinsimpleformula}
\end{center}
\end{figure}

This looks complicated, but look at the diagran shown in \ref{fig:flowinsimpleformula}.
There are some tricks in the code: the \verb%x% and \verb%y% sources are reused, 
which is only safe because they're constant sources, and the \verb%add%, \verb%sub%,
and \verb%div% chips are one shots.

Exercise (Hard). If we wanted to make this system accept a list of pairs and process them,
printing the values of x and y and the quotient, or an error message,
what would we need to do?


\chapter{Advantage of coroutines}
You may wonder why bother with coroutines? What's wrong with
ordinary functions and procedures?

The answer is: in the right context, functions and procedures
are very useful. But they're a lot weaker than you think.
Coroutines are to be treated as another technique, not a replacement
for other techniques.

We will exhibit a critical case which shows beyond doubt
that your conceptions about how great functional programming
is are completely misplaced. Functional programming is great for
functions but does not work so well when dealing with non termination
or partial functions. In fact, it is so weak that the so called
functional programming paradigm can be considered totally discreted
along with object orientation.

These system has a shared fault: the subroutine. Subroutines involve
a master slave relationship which skews your program design one
way or another, and no way is natural. Coroutines fix this problem
so you only use subroutines when they're natural. Coroutines provide
a peer to peer relationship when that is the best way to do things.

\section{Folds}
The example I will use requires you to pretend that something simple
could be more complicated and to envisage what that entails. 
I am going to use the classic functional programming function,
the fold and show that functional programming is evil, and fold is 
perfect example of what is wrong with functional programming!

\subsection{List Folds}
First lets see a list in Felix:

\begin{minted}{felix}
union list[Element] =
  | Empty 
  | Cons of Element * list[Element]
;
\end{minted}

Now a top down, or left fold:

\begin{minted}{felix}
fun fold_left[Element, State] 
  (acc: Element->State->State])
  (init: State)
  (ls: list[Element])
=>
  match ls with
  | Empty => init
  | Cons (head, tail) =>
    fold_left acc (acc head init) tail
;
\end{minted}

I have written the routine in a functional style using recursion,
it is in fact tail recursive. A right fold starts from the other
end of the list and traditionally looks like this:

\begin{minted}{felix}
fun fold_right[Element, State] 
  (acc: State->Element->State)
  (ls: list[Element])
  (init: State)
=>
  match ls with
  | Empty => init
  | Cons (head, tail) =>
    acc (fold_right acc tail init) head
;
\end{minted}

It is not tail recursion. Instead, we recurse down to the end of
the list and fold the elements in to the result as we pop back up.

\subsection{Tree Folds}
A binary search tree has more useful orderings. 

\begin{minted}{felix}
union tree [Element] =
  | Leaf
  | Node of Element * tree[Element] * tree[Element]
;
\end{minted}

All tree visitors in a functional setting use a recursion, 
however the order of visiting elements is determined by when the 
client accumulator is called. Prefix order, or left most depth first is the easiest:

\subsection{Pre-order Fold}
This fold visits the deepest left most element first.

\begin{minted}{felix}
fun preorder [Element, State]
  (acc: Element -> State -> State)
  (init: State)
  (tr: tree[Element])
=>
  match tr with
  | Leaf => init
  | Node (elt, left, right) =>
    let v = acc elt lv in
    let lv = preorder acc init left in
    preorder acc v right
;
\end{minted}

\subsection{In-order Fold}

\begin{minted}{felix}
fun inorder [Element, State]
  (acc: Element -> State -> State)
  (init: State)
  (tr: tree[Element])
=>
  match tr with
  | Leaf => init
  | Node (elt, left, right) =>
    let lv = inorder acc init left
    let v = acc elt lv in
    inorder acc v right in
;
\end{minted}


\subsection{Post-order Fold}

\begin{minted}{felix}
fun postorder [Element, State]
  (acc: Element -> State -> State)
  (init: State)
  (tr: tree[Element])
=>
  match tr with
  | Leaf => init
  | Node (elt, left, right) =>
    let lv = postorder acc init left
    let rv = postorder acc lv right in
    acc elt rv in
;
\end{minted}

\subsection{Breadth First Fold}
This is much harder to do efficiently. 
A simple routine is not so hard, lets start
with a descent which folds a single level,
it returns the folded value and a flag which
tells if that level is populated:

\begin{minted}{felix}
fun bfn[Element, State]
  (acc: Element -> State -> State)
  (init: State)
  (tr: tree[Element])
  (n:int)
=>
  match tr with
  | Leaf => init, false
  | Node (elt, left, right) =>
    if n > 0 then
      let lv,lflag = bfn acc init left (n - 1) in
      let rv, rflag = bfn acc lv right (n - 1) in
      rv, lflag or rflag
    else 
      acc elt init, true
    endif
;
\end{minted}

The cost of this algorithm is roughly as follows: there are $2^k$ elements
in level $k$ where k starts at 0, assuming a fully populated balanced
tree of sufficient depth, so we have to scan $\Sigma_{k=0}^n 2^k$
elements, including the target level $n$, but this is known
to be just $2^{n+1}-1$.

Now the idea is simply to fold level 1, then level 2, 
then level 3, etc, until the flag tells us we hit an
entirely unpopulated level: this is easier to do
imperatively but we want a functional solution:


\begin{minted}{felix}
fun bfaux[Element, State]
  (acc: Element -> State -> State)
  (init: State)
  (tr: tree[Element])
  (n:int)
=>
  let v,flag = bfn acc init tr n in
  if flag then 
     bfaux acc init tr (n + 1)
  else 
    v
  endif
;

fun bf[Element, State]
  (acc: Element -> State -> State)
  (init: State)
  (tr: tree[Element])
=>
  bfaux acc init tr 0
;
\end{minted}

Now the total cost of folding a fully populated balanced tree to level $n$
is $\Sigma_{k=0}^{n+1} (2^k-1)$ where we had to use $n+1$ to account for the
fully empty level below which has to be scanned for the termination check.
We can throw out the $-1$ here, and the result is just $2^{n+2}$ which,
suprisingly, is only four times the number of elements in the tree. 
The routine isn't as inefficient as you might think, and it has
a major advantage of other routines: it uses no auxilliary storage,
and the use of the stack is limited to some small multiple of the
tree depth, which is insignificant if the tree is reasonable well
shaped.


\section{Iterators}
Another way to visit values in a data structures is to use iterators.
We will show iterators corresponding to the folds above in a style
similar to what would be required in C++, however we will use a single
get method to get the next value which returns an option type, so
that the end of the data stream can be detected. We'll also use a purely
functional style.

\subsection{Left List Iterator}
The left visitor is quite easy, we use a list of the same type
as the input as the state:

\begin{minted}{felix}
fun first_llit[Element] 
  (ls: list[Element])
:
  opt[Element] * list[Element] 
=> 
  match ls with
  | Empty => None[Element], Empty[Element]
  | Cons (head, tail) => Some head, tail
;

fun next_llit[Element] 
  (ls: list[Element]) 
:
  opt[Element] * list[Element] 
=> 
  first_llit ls
;
\end{minted}


\subsection{Right List Iterator}
The bottom up iterator is harder:

\begin{minted}{felix}
fun first_rlit[Element] 
  (ls: list[Element]) 
:
  opt[Element] * list[Element] 
=> 
  first_llit (rev ls)
;

fun next_rlit[Element] 
  (ls: list[Element]) 
:
  opt[Element] * list[Element] 
=> 
  first_llit ls
;
\end{minted}

The code is simple, it just uses the left iterator
on a reversed list. There is an overhead in space and
time which existed for the functional fold as well.
The difference is, the iterator must maintain the state
explicitly on the heap, whereas the fold uses the machine
stack to hold pointers into the list implicitly.

\subsection{Left Tree Iterator}
Now the real fun starts! How must our iterator work?

\begin{minted}{felix}
union Todo[Element] = 
  | Value of Element
  | Tree of tree[Element]
;
typedef zipper[Element] = list[Todo[Element]];

obj Iterata[Element] (tr: tree[Element]) =
{
  var path = Empty[Todo[Element]];
  setup (tr);

  proc setup (t: tree[Element]) {
    match t with
    | Leaf => return;
    | Node (elt, left, right) =>
      path = Cons (Tree right, path);
      path = Cons (Value elt, path);
      setup left;
    endmatch; 
  }

  method gen next () => {
    match path with
    | Empty => return None[Element];
    | Cons (Value v, tail) =>
      path = tail;
      return Some v;
    | Cons (Tree t, tail) =>
      path = tail;
      setup t;
      return next();
    endmatch;
  }
}
\end{minted}

As you can see, this is considerably more complicated than the corresponding fold.
In fact, it took me a couple of minutes to write the fold and many hours
until I figured out how to write the iterator. It is basically the
{\em control inverse} of the fold: it is the fold turned inside out.

Control inversion is a key concept. The fold function is a {\em master} which
calls the client function as a {\em slave}. The iterator, on the other hand,
is a slave function called by the client, which is the master.

The iterator above reveals the true nature of the data required by a preorder
tree visitation: the zipper above represents a type which can hold the state.
In the fold the use of the machine stack, recursion, and local variables
hides the zipper: it uses the machine stack to couple the program counter with
the local data. The iterator cannot do that, since it loses the stack each
call. Instead it manually maintains its own stack, the path value.

The iterator above is only an input iterator. We can translate the mutations
to produce a forward iterator by using a monadic form:

\begin{minted}{felix}
fun setup(t: tree[Element]) (path: zipper[Element]) =>
  match t with
  | Leaf => path
  | Node (elt, left, right) => 
    let path2 = Cons (Tree right, path) in
    let path3 = Cons (Value elt, path2) in
    setup left path3
;

fun next (path: zipper[Element]) =>
  match path with
  | Empty => None[Element], Empty[Todo[Element]]
  | Cons (Value v, tail) =>
    Some v, tail
  | Cons (Tree tr, tail) =>
    let path2 = setup tail path in
    next path2
;
\end{minted}

\section{Zippers}
For any inductive data type, there is a related type known as a {\em zipper}.
A zipper is basically a path in the tree. It can be thought of as the
original data type with a hole in it representing the location of
the current visitor, that is, a way to cut off a branch forming a subtree
and a tree with a missing branch.

If we use the pure form of a tree it is a given by the formula:

\begin{minted}{felix}
typedef tree[T] = 1 + T * tree[T] * tree[T];
\end{minted}

where \verb%+% is the infix operator for an anonymous union
or sum type, and $1$ is the unit type. This has the form of a polynomial

\[1 + T X^2\]

which has the derivative

\[2TX\]

or

\begin{minted}{felix}
typedef zipper[T] = bool * T * tree[T]
\end{minted}

In this form the boolean value is used to decide whether
to process the value term next, or the right tree branch.
In the zipper I presented, I split these two cases up into
two list components \verb%Value% and \verb%Tree% and put them
in the desired order in advance, because that was easier 
to understand than presenting both with a selector. If the 
selector is false, we process the value and set it to true,
if it is true we process the tree and then discard the zipper
node: in my implementation the value is first on the stack,
then comes the tree, so the correct order is obtain by simply
popping each element from the zipper as it is processed.

What's critical again is that the fold and iterator both maintain
the same data that the zipper specifies, although the encoding
is quite different, the fold maintaining it entirely implicitly.

Functional code is not always easier! If you consider a breadth
first ordering, then both the functional and iterator versions
must manually maintain some state. A simple functional breadth first fold could be 
done with a recursive descent and a depth limit, so all the values at a given
depth are processed, then all the values one level deeper, and so on.
Arranging termination is not entirely trivial, but could be done by simply
calculating the maximum depth, again using a recursive descent. 
The fold itself would pass over intermediate level nodes several times,
indeed in the degerate case of a list the top node would be scanned
N times, for a list of length N, and the overall performance would be
quadratic.

A better algorithm could avoid rescanning by keeping track better.
This can be done by accumulating a list of all the children, reversing
it and scanning it processing its values, and building up the list
for the next pass. This handles termination correctly however the
performance advantage is questionable since the list has to be constructed
which takes time and uses up space: when the list is discarded it
loads up the garbage collector. A simple recursive descent in a balanced
binary tree only doubles the cost, because the number of ancestors of
a set of siblings is always exactly equal to the number of siblings
minus one.

\section{The Client Code}
I have shown the fold and control inverted fold, the iterator,
so you can see clearly that in general the fold is superior because
it is simpler. It can use recursion and call the client code wherever
and whenever it wants. The iterator form is at a severe disadvantage
because it is a callback or slave subroutine.

It is vital to understand that this imbalance is not because functional
programming is better: the iterator form has a monadic functional
equivalent, which is just as complicated as the procedural form,
if not more so: the advantage of the functional form is that it 
produces a forward iterator, the disadvantage is that the client
code must maintain the state, which in our examples is the path
representing the derivative.

What you must now see is that if you use a master fold, your client
code has a problem: it is a slave: the client argument of a fold
function is a callback, and if it is to do something complicated,
it must manually maintain state. It cannot use local variables,
recursion, or the machine stack.

On the other hand, the iterator form rules supreme for the client
code. It can use recursion and the machine stack, and simply call
the iterator whenever and wherever it wants for the next value.
If you're using the functional forward iterator, you have to pass
the state value around, but this is equivalent to being able
to access the iterator object in the procedural form.

You will need some imagination to see that whilst in general
state is required with both the fold and iterator methods,
the use of manually constructed data structures on the heap
may be necessary in the functional form 
if the linear machine stack is inadequate. Iterator clients
are not necessarily trivial!

However the superiority of the iterator form .. for the client
programmer .. is easy to demonstrate unequivocably by considering
a really simple and very old algorithm: the merge.

A merge takes two sorted lists and merges them into a single
sorted list. The algorithm is simple: look at the head of each
list and pop off the smallest element, push it onto another list.
Keep going until both lists are empty and reverse the result list.

Simple, and easy to make purely functional but there's a minor
problem. Its simple if the client code can choose which list to pop.
So its simple if you have two iterators! 

In fact with iterators .. the same algorithm works, even if the data
is coming from a tree instead of a list, because the iterator is
converting the data structure into a value stream in all cases.

So how do you do a merge using a pair of folds?

Er .. well you can't. The iterators are clearly and unequivocably
superior. The writer of a fold has an advantage. The client has
a disadvantage. In this case, the disadvantage is a killer.

The way to do this is run the two folds separately to make two lists
and then use then to feed your client code. If your language is lazily
evaluated this is not necessarily inefficient but now we're exposing
a well known major weakness of functional programming: the performance
of your algorithms depends heavily on your compiler and run time
implementation.

It is indeed interesting that lazy evaluation may allow the suspension
of two folds over two trees whose client code produces two lists which
are then consumed by a merge. [More needs to be said here!]

\section{Solving the problem with coroutines}
It is critical at this point to understand there is a problem.
Fold simply doesn't work, the client code is too hard to write!
On the other hand iterator client code is easy to write, but
the iterators are too hard to write!

Neither method is any good! Both suffer from the same problem:
slavery! Slave subroutines do not implicitly retain state on the
machine stack synchronised with control flow as masters do.
Masters are better! 

What's the answer? It's pretty obvious, we need two stacks, and we
need two masters. But you cannot do that with mere subroutines.
You cannot do it with traditional procedural code nor with
functional code. Procedures and functions are both subroutines.

Since you read the title of this paper you already know the answer:
coroutines. Coroutines cooperate as peers. They're constructed as if they're
masters so both the visitor of the data structure and client code are
easy. In fact we shall soon see, the data structure driver code in
coroutine form looks exactly like the superior functional fold,
and the client code looks exactly like the superior iterator client code.
With coroutines, we can capture the best of both worlds!

\section{Coding Visitors with coroutines}
We're now going to attack the tree fold problem using coroutines.
Recall the depth first functional routine:

\begin{minted}{felix}
fun preorder [Element, State]
  (acc: Element -> State -> State)
  (init: State)
  (tr: tree[Element])
=>
  match tr with
  | Leaf => init
  | Node (elt, left, right) =>
    let lv = preorder acc init left in
    let v = acc elt lv in
    preorder acc v right
;
\end{minted}

Now see the coroutine, this is a full working example.
First some test data:

\begin{minted}{felix}
union tree [Element] =
  | Leaf
  | Node of Element * tree[Element] * tree[Element]
;

var t = 
  Node (10,
    Node (5,
      Node (1, Leaf[int], Leaf[int]),
      Node (7, Leaf[int], Leaf[int])
    ),
    Node (15,
      Node (12, Leaf[int], Node (13,Leaf[int], Leaf[int])),
      Node (17, Node (16, Leaf[int], Leaf[int]), Leaf[int])
    )
  )
;
\end{minted}

and now the coroutine:


\begin{minted}{felix}
proc copreorder [Element]
 (out: %>Element)
 (tr: tree[Element])
{
  match tr with
  | Leaf => return; 
  | Node (elt, left, right) =>
    copreorder out left;
    write (out, elt);
    copreorder out right;
  endmatch;
}
\end{minted}

and now we need a way to check the results:

\begin{minted}{felix}
proc printer (ch: %<int) { println$ read ch; printer ch; }

begin
  var inp, out = mk_ioschannel_pair[int]();
  spawn_fthread { copreorder out t; };
  spawn_fthread { printer inp; };
end
\end{minted}


\part{Appendices}
\chapter{Coroutine Semantics}
\label{Coroutine Semantics}
\section{Objects}
A coroutine system consists of the following types of objects:
\begin{description}
\item[Scheduler] A device to hold a set of active fibres and
select one to be current.
\item[Channels] An object to support synchronisation and data transfer.
\item[Fibres] A thread of control which can be suspended and resumed.
\item[Continuations] An object representing the future of a coroutine.
\end{description}

\subsection{Scheduler States}
A scheduler is in one of two states:
\begin{description}
\item[Current] The currently running scheduler
\item[Suspended] A scheduler for which the Running fibre is
executing another scheduler.
\end{description}

\subsection{Fibre States}
Each fibre is in one of these states:
\begin{description}
\item[Running] Exactly one fibre per scheduler is always running.
\item[Active] Fibes which are ready to run but not running on a particular scheduler.
\item[Hungry] Fibres suspended waiting for input on a channel.
\item[Blocked] Fibres suspended waiting to perform output on a channel.
\end{description}

\subsection{Channel States}
Each channel is in one of these states:
\begin{description}
\item[Empty] There are no fibres associated with the channel.
\item[Hungry] A set of hungry fibres are waiting for input on the channel.
\item[Blocked] A set of blocked fibres are waiting to perform output on the channel.
\end{description}


\section{Abstract State}
\subsection{State Data by Sets}
A fibration system consists of 
\begin{enumerate}
\item A set of fibres $\mathcal F$
\item A set of channels $\mathcal C$
\item An integer $k$
\item An indexed set of schedulers $\mathcal S= \{s_i\}{\rm\ for\ }i=1 {\rm\ to\ }k$  
\end{enumerate}
and the following relations:
\begin{enumerate}
\item for each $i=1 {\rm\ to\ }k$ a pair $(R_i, \mathcal A_i)$ where $R_i$ is a fibre
and $\mathcal A_i$ is a set of fibres, these fibres being associated with
scheduler $s_i$, $R_i$ is the currently Running fibre of the scheduler,
and $\mathcal A_i$ is the set of Active fibres;
\item for each channel $c$ a set $\mathcal H_c$ of Hungry fibres
and a set $\mathcal B_c$ of Blocked fibres, such that one of these sets
is empty, if both sets are empty, the channel is said to be Empty,
otherwise it is said to be Hungry or Blocked depending on whether
the Hungry or Blocked set is nonempty;

\item A reachability relation to be described below
\end{enumerate}
with the requirement that each fibre is in precisely one of the sets $\{R_i\}$,
$A_i$, $H_c$ or $B_c$.

We define the relation 
\begin{align}
\mathrel H &= \{(f,c) \mid f \in \mathrel H_c\}&\rm Hunger\\
\mathrel B &= \{(f,c) \mid f \in \mathrel B_c\}&\rm Blockage\\
\mathcal F_{\mathcal H} &= \{f \mid \exists c . (f,c) \in \mathcal H\}&\rm Hungry\ Fibres\\
\mathcal F_{\mathcal B} &= \{f \mid \exists c . (f,c) \in \mathcal B\}&\rm Blocked\ Fibres\\
\mathcal C_{\mathcal H} &= \{ c \mid \exists f . (f,c) \in \mathcal H\}&\rm Hungry\ Channels\\
\mathcal C_{\mathcal B} &= \{ c \mid \exists f . (f,c) \in \mathcal B\}&\rm Blocked\ Channels\\
\mathcal E &= \{c \mid | \mathcal H_c = \mathcal B_c = \emptyset \}&\rm Empty\ Channels
\end{align}

\subsection{State Data by ML}
Using an ML like description may make the state data easier
to visualise.

\begin{minted}{felix}
scheduler =
  Run: fibre | NULL, 
  Active: Set[fibre]

channel = 
  | Empty 
  | Hungry: NonemptySet[fibre]
  | Blocked: NonemptySet[fibre]

fibre = (current: continuation)

continuation = 
  caller: continuation | NULL,
  PC: codeaddress,
  local: data
\end{minted}

\section{Operations}

\subsection{Spawn} 
The spawn operation takes as an argument a unit procedure and makes
a closure thereof the initial continuation
of a new fibre.  Of the pair consisting of the currently running
fibre (the spawner) and the new fibre (the spawnee) one will have Active
state and the other will be Running. It is not specified which
of the pair is Running.

\begin{equation}
{\mathcal F} \leftarrow {\mathcal F} \cup \{f\}
\end{equation}

where f is a fresh fibre and

\begin{equation}
R_k,{\mathcal A_k} \leftarrow
\begin{cases}
R_k,{\mathcal A_k} \cup \{f\} \\
f,{\mathcal A_k} \cup \{R_s\} \\
\end{cases}
\end{equation}

where the choice between the two cases is indeterminate.

\subsection{Run} 
The run operation is a subroutine. It increments $k$ and creates
a new scheduler $s_k$. The scheduler $s_{k-1}$ is Suspended. 

\begin{equation}
k \leftarrow k + 1
\end{equation}


It then takes as an argument a unit procedure and makes
a closure thereof the initial continuation
of a new fibre $f$ and makes that the running fibre $R_k$ of
the new current scheduler. The set of active fibres $A_k$
is set to $\emptyset$. 

 
\begin{equation}
{\mathcal F} \leftarrow {\mathcal F} \cup \{f\}
\end{equation}

where f is a fresh fibre and

\begin{equation}
R_k,\mathcal A_k \leftarrow
f,\emptyset
\end{equation}

The scheduler is then run as a subroutine. It returns when there
is no running fibre, which implies also there are no active
fibres left. $k$ is then decremented, scheduler $s_k$ again becomes Current, 
and the the
current continuation of its running fibre resumes.
\begin{equation}
k \leftarrow k - 1
\end{equation}

\subsection{Create channel}
A function which creates a channel.

\begin{align}
{\mathcal C} \leftarrow {\mathcal C} \cup \{c\}\\
{\mathcal E} \leftarrow {\mathcal E} \cup \{c\}
\end{align}
where c is a fresh channel.

\subsection{Read}
The read operation from fibre $r$ takes as an argument a channel $c$.

\begin{enumerate}
\item If the channel is Empty, the Running fibre performing the read
changes state to Hungry, the channel changes state to Hungry,
and the fibre is associated with the channel.
\begin{align}
{\mathcal H}&\leftarrow {\mathcal H} \cup \{(r,c)\}\\
{\mathcal E} &\leftarrow {\mathcal E} \setminus \{c\}
\end{align}

If there are no active fibres, the program terminates,
otherwise the scheduler selects an Active fibre and
changes its state to Running.  It is not specified which active 
fibre is chosen.

\begin{equation}
R_k,{\mathcal A_k} \leftarrow
\begin{cases}
\epsilon,{\mathcal A_k}& {\rm if\ } A_k=\emptyset \\
a,{\mathcal A_k} \setminus \{a\} & {\rm some\ } a\in A
\end{cases}
\end{equation}


\item If the channel is Hungry, the Running fibre changes state
to Hungry, and the fibre is associated with the channel.

\begin{align}
{\mathcal H}&\leftarrow {\mathcal H} \cup \{(r,c)\}\\
\end{align}

If there are no active fibres, the program terminates,
otherwise the scheduler selects an Active fibre and
changes its state to Running.  It is not specified which active 
fibre is chosen.

\begin{equation}
R_k,{\mathcal A_k} \leftarrow
\begin{cases}
\epsilon,{\mathcal A_k}& {\rm if} A_k=\emptyset \\
a,{\mathcal A_k} \setminus \{a\} & {\rm some\ } a\in A_k
\end{cases}
\end{equation}


\item If the channel is Blocked, one of the associated Blocked fibres $w$
is selected, and dissociated from the channel. 

\begin{equation}
{\mathcal B} \leftarrow {\mathcal B}\setminus (w,c)\\ 
\end{equation}

Of these two
fibres, one is changed to state Active and the other to Running.
It is not specified which fibre is chosen to be Running.

\begin{equation}
R_k,{\mathcal A_k} \leftarrow
\begin{cases}
R_k,{\mathcal A_k} \cup \{w\} \\
w,{\mathcal A_k} \cup \{R\} \\
\end{cases}
\end{equation}

The value supplied to the write operation of the Blocked
fibre will be pass to the Hungry fibre when it transitions
to Running state.


\end{enumerate}



\subsection{Write}
The write operation performed by fibre $w$ takes two arguments, a channel and a value
to be written.

\begin{enumerate}
\item If the channel is Empty, the Running fibre performing the write
changes state to Blocked, the channel changes state to Blocked,
and the fibre is associated with the channel.
\begin{align}
{\mathcal B}&\leftarrow {\mathcal B} \cup \{(w,c)\}\\
{\mathcal E} &\leftarrow {\mathcal E} \setminus \{c\}
\end{align}


If there are no active fibres, the program terminates,
otherwise the scheduler selects an Active fibre and
changes its state to Running.  It is not specified which active 
fibre is chosen.

\begin{equation}
R_k,{\mathcal A_k} \leftarrow
\begin{cases}
\epsilon,{\mathcal A_k}& {\rm if\ } A_k=\emptyset \\
a,{\mathcal A_k} \setminus \{a\} & {\rm some\ } a\in A
\end{cases}
\end{equation}


\item If the channel is Blocked, the Running fibre changes state
to Blocked, and the fibre is associated with the channel.

\begin{align}
{\mathcal B}&\leftarrow {\mathcal B} \cup \{(w,c)\}\\
\end{align}


If there are no active fibres, the program terminates,
otherwise the scheduler selects an Active fibre and
changes its state to Running.  It is not specified which active 
fibre is chosen.

\begin{equation}
R_k,{\mathcal A_k} \leftarrow
\begin{cases}
\epsilon,{\mathcal A_k}& {\rm if\ } A_k=\emptyset \\
a,{\mathcal A_k} \setminus \{a\} & {\rm some\ } a\in A
\end{cases}
\end{equation}


\item If the channel is Hungry, one of the associated Hungry fibres $r$
is selected, and dissociated from the channel.

\begin{equation}
{\mathcal H} \leftarrow {\mathcal H}\setminus (r,c)\\ 
\end{equation}


Of these two
fibres, one is changed to state Active and the other to Running.
It is not specified which fibre is chosen to be Running.

\begin{equation}
R_k,{\mathcal A_k} \leftarrow
\begin{cases}
R_k,{\mathcal A_k} \cup \{r\} \\
r,{\mathcal A_k} \cup \{R\} \\
\end{cases}
\end{equation}


The value supplied by the write operation of the Blocked
fibre will be pass to the Hungry fibre when it transitions
to Running state.
\end{enumerate}

\subsection{Reachability}
The Running, and, each Active fibre and its associated call chain of 
continuations are deemed to be Reachable.

If a channel is known to reachable fibre, it is also reachable.
A channel may be known because its address is stored in the local
data of a continuation of a fibre, or, it is reachable via some object
which can be reached from local data. The exact rules are 
programming language dependent.

Each fibre associated with a reachable channel is reachable.

The transitive closure of the reachability relation consists
of a closed, finite, collection or channels and fibres which
are reachable.

Unreachable fibres and channels are automtically garbage
collected.

\subsection{Elimination}
Fibres and channels are eliminated when they are 
no longer reachable.

A fibre may become unreachable in three ways.

\subsubsection{Suicide}
A fibre for which the initial continuation returns is said to be
dead, and  becomes unreachable. If there are no longer any Active fibres,
the program returns, otherwise the scheduler picks
one Active fibre and changes its state to Running.

\subsubsection{Starvation}
A fibre in the Hungry state becomes unreachable when the
channel on which it is waiting becomes unreachable.

\subsubsection{Blockage}
A fibre in the Blocked state becomes unreachable 
when the channel on which it is waiting becomes unreachable.

\section{LiveLock}
If a fibre is Hungry (or Blocked) on a reachable channel
but no future Running fibre will write (or read) that
channel, the fibre is said to be livelocked. The fibre
will never proceed but it cannot be removed from
the system because it is reachable via the channel.

A livelock is considered to transition to a deadlock
if the channel becomes unreachable, in which case
the fibre will becomes unreachable and is said to
die through Starvation (or Blockage),
disolving the deadlock. In other words, fibres cannot deadlock.


\section{Fibre Structure}
Each fibre consists of a single current continuation.
Each continuation may have an associated continuation
known as its caller. The initial continuation of a freshly
spawned fibre has no caller.

The closure of the caller relation leads to a linear
sequence of continuations starting with the current
continuation and ending with the initial continuation
of a freshly spawned fibre.

The main program consists of an initially Running
fibre with a specified initial continuation.

Continuations have the usual operations of a procedure.
They may return, call another procedure, spawn new
fibres, create channels, and read and write channels,
as well as the other usual operations of a procedure
in a general purpose programming language.

A continuation is reachable if it is the current
continuation of a reachable fibre, or the caller
of a reachable continuation.

A continuation is formed by calling a procedure,
which causes a data frame to be constructed which
contains the return address of the caller,
parameters and local variables of the procedure,
and a program counter containing the current
locus of control (code address) within the procedure.
The program counter is initially set to the specified
entry point of the procedure.

A coroutine is a procedure which directly or indirectly
performs channel I/O. Coroutines may be called by 
other coroutines, but not by procedures or functions.
Instead, a coroutine may be spawned by a procedure,
or run by a procedure or function. This creates a
fibre which hosts the created continuation.

Note: the set of fibres and channels created directly
or indirectly by a run subroutine called inside
a function should be isolated from all other fibres
and channels to ensure the function has no side-effects.



\section{Continuation Structure}
\subsection{Continuation Data}
A continuation has associated with it the following
data:

\begin{description}
\item[caller] Another continuation of the same fibre which is
suspended until this continuation returns.
\item[data frame] Sometimes called the stack frame, contains
local variables the continuation may access.
\item[program counter] A location in the program code representing
the current point of this continuations execution or suspension
\end{description}

\subsection{Continuation operations}
The current continuation of a fibre executes a wide range of
operations including channel I/O, spawning new fibres,
calling a procedure, and returning.

\begin{description}
\item[call] Calling a procedure creates a new continuation
with its program counter set at the procedure entry point,
and a fresh data frame. The new continuation becomes the
current continuation, the current continuation suspends.
The new continuations caller field is set to the caller.
The current continuation program counter is set
to the pointer after the call instruction.

The effect is push an entry onto the fibres continuation chain.

\item[return] Returning from the current continuation causes
the owning fibres current continuation to be set to the
current continuations caller, if one exists, or the 
fibre to be marked Dead if there is no caller. Execution
of the suspended caller continues at its program counter.

The effect is to pop an entry off the fibre's continuation chain.

\item[read/write] Channel I/O suspends the current continuation
of a fibre until a matching operation from another fibre
synchronises with it. A read is matched by a write, and a write
is matched by a read.
\end{description}

By the rules of state change, channel I/O should be viewed
as performing a peer to peer neutral exchange of control:
the current fibre becomes suspended without losing its position
and hands control to another fibre. Later, control is handed
back and the fibre continues.

Coroutine based systems, therefore, operate by repeated exchanges
of control accompanied by data transfers in a direction independent
of the control flow, which sets coroutines aside from functions.

\section{Events}
Each state transfer of the fibration system may be considered
an event. However the key events are
\begin{itemize}
\item spawning
\item suicide
\item entry to a read operation
\item return from a read operation
\item entry to a write operation
\item return from a write operation
\end{itemize}

I/O synchronisation consists of suspension on entry
to a read or write operation, and simultaneously release
of suspension, or resumption, on matching write or read.

I/O suspension occurs when a fibre becomes Hungry or Blocked,
and resumption when it becomes Running or Active.

Fibrated systems are characterised by a simple rule: events are
totally ordered. The order may not be determinate. 

\section{Control Type}
The control type of a coroutine is defined as follows.
We assume the coroutine is spawned as a fibre, and each
and every read request is satisfied by a random value of
the legal input type. Write requests are also satisfied.
We cojoin entry and return from read into a single read
event, and entry and return from write into a single
write event, since we are only interested in the behaviour
of the fibre.

The sequence of all possible events which the fibre
may exhibit is the coroutines control type. Note, the
control type is a property of the coroutine (procedure).

\section{Encoding Control Types}
In general, the control type of a coroutine can be quite
complex. However for special cases, a simple encoding
can be given.

\subsection{One shots}
A one-shot coutine is one that exhibits a bounded number
of events before suiciding. The three most common one
shots are:

\begin{description}
\item[Value: type W] A coroutine which writes a single value to
a channel and then exits.
\item[Result: type R] A coroutine which reads a single value to
from channel and then exits.
\item[Function: type RW] A couroutine which reads one value
from a channel, calculates an answer, writes that
down a channel and then exits.
\end{description}

\subsection{Continuous devices}
A continuous coroutine is one which does not exit.
It can therefore terminate only by starvation or blockage.
The three most common kinds of such devices are
\begin{description}
\item[Source: type W+] Writes a continuous stream of values
to a channel.
\item[Sink: type R+] Reads a continuous stream of values
from a channel.
\item[Transducer] Reads and writes.
\end{description}

Because the sequence of events is a stream, we may use
convenient notations to describe control types.
If possible, a regular expression will be used.
Sometimes, a grammar will be required. In other cases
there is no simple notation for the behaviour of a coroutine.

We will use postfix \verb%+% for repetition.
 
\subsection{Transducer Types}
A transducer which read a value, write a value, then loops
back and repeats is called a {\em functional transducer},
it may be given the type (RW)+.

In a functional language, a partial function has no natural
encoding. There are two common solutions. The first is to
return an option type, say Some v, if there is a result,
or None if there is not. This solution involves modifying
the codomain. The other solution is to restrict the domain so
that the subroutine is a function.

Coroutines, however, represent partial functions naturally.
If a value is read for which there is no result, none is written!
The type of a {\em partial function transducer} is therefore
given by ((R+)W)+, in other words multiple reads may occur
for each write. Note that two writes may not occur in succession.

This type may also be applied to many other coroutines,
for example the list filter higher order function.

\subsection{Duality}
Coroutines are dual to functions. The core difference is that
they operate in time not space. Thus, in the dual space
a spatial product type becomes a temporal sequence.

Coroutines are ideal for processing streams. Whereas
function code cannot construct streams without laziness,
and cannot deconstruct them without eagerness, coroutines
are neither eager nor lazy.

One may view an eager functional application as driving a
value into a function to get a result, and a lazy application
as pulling a value into a function. Pushing value implies
eagerly evaluating it, pulling implies the value is calculated
on demand.

Coroutine simultaneously push and pull values across channels
and so eliminate the evaluation model dichotomy that plagues
functional programming. This coherence does not come for
free: it is replaced by indeterminate event ordering.

\section{Composition}
By far the biggest advantage of coroutine modelling is the 
ultimate flexibility of composition. Coroutines provide
far better modularity and reusability than functions,
but this comes at the price of complexity. You will observe
considerably more housekeeping is required to compose
coroutines than procedures or functions, because, simply,
there are more way to compose them.

A collection of coroutines can be regarded as black boxes
resembling chips on a circuit board, with the wires
connecting pins representing channels. So instead of using
variables and binding constructions, we can construct more
or less arbitrary networks.

\subsection{Pipelines}
The simplest kind of composition is the pipeline. It is a sequence
of transduces wired together with the output of one transducer
connected by a channel to the input of the next.

If the pipeline consists entirely of transducers is is an open
pipeline. If there is a source at one end and a sink at the other
it is a closed pipeline. Partially open pipelines can also exist.

The composition of two transducers has a type dependent on
the left and right transducer types. 

With a functional transducer, you would expect the composition
of (R1W1)+ with (R2W2)+ to be (R1W2)+ but this is not the case!

Consider, the left transducer performs R1, then W1, then right
performs R2. At this point it is not determinate whether left
or right proceeds. If left proceeds, we have R1 again, then W1.
then right proceeds and performs W2 before coming back to read R2,
and what happens next is again indeterminate. The sequence is
therefore R1, W1/R2, R1, W1/R2 which shows R1 can be read twice
before W2 is observed. We have written w/r here to indicate synchronised
events which are abstracted away when describing the observable
behaviour of the composite.

Clearly, (R1?R1W2?W2)+ contains the set of possible event sequences,
but then (R1+R2+)+ contains it, and therefore the set of possible
event sequences as well. So we should seek the most precise, or
{\em principal} type of the composite.

We can calculate the type from the operational semantics.
At any point in time, the system must be in one of a finite
number of states. Where we have indeterminacy, the transitions
out of a given state are not fully specified. The result is
clearly a non-deterministic finite state automaton.

We must observe, such an automaton corresponds to (one or more) 
larger deterministic finite state automata. This is an important
result because it has practical implications: it means we can
pick a DFA and use it to optimise away abstracted synchronisation
points. In other words, we build a fast model of the system
by inlining and using shared variables instead of channels,
and then eliminate the variables by functional composition.

This is the primary reason we insist on indeterminate behaviour:
it allows composition to be subject to a reduction calculus.

\section{Felix Implementation}
The following functions and procedures are provided in Felix:

\begin{minted}{felix}
spawn_fthread: (1 -> 0) -> 0;
run: (1 -> 0) -> 0;
mk_ioschannel_pair[T]: 1 -> ischannel[T] * oschannel[T];
read[T]: ischannel[T] -> T
write[T]: oschannel[T] * T -> 0
\end{minted}

In the abstract, channels are bidirectional and untyped.
However we will restrict our attention to channels
typed to support either read (ischannel) or write (oschannel)
of a value of a fixed data type.

The following shorthand types are available:

\begin{minted}{felix}
%<T    ischannel[T]
%>T    oschannel[T]
\end{minted}

More advanced typing exploiting channel capabilities are
discussed later.

Simple example program:

\begin{minted}{felix}
proc demo () {
  var inp, out = mk_ioschannel_pair[int]();

  proc source () {
    for i in 1..10 perform write (out,i);
  }

  proc sink () {
    while true do
      var j = read inp;
      println$ j;
    done
  }

  spawn_fthread source;
  spawn_fthread sink;
}
demo();
\end{minted}

In this program, we create a channel with an input and
output end typed to transfer an int. The source coroutine
writes the integers from 1 through to 10 inclusive to
the write end of the channel, the sink coroutine
reads integers from the channel and prints them.

The main fibre calls the demo procedure which launches
two fibres with initial continuations the closures of
the source and sink procedures.

When demo returns, the main fibre's current continuation
no longer knows the channel, so the channel is not reachable
from the main fibre. 

The source coroutine returns after sending 10 integers to
the sink via the channel.  When a fibre no longer has
a current continuation, returning to the non-existent
caller causes the fibre to no longer have a legal
state. This is known as suicide.

After the sink has read the 
last value, it becomes permanently Hungry. The sink 
procedure dies by starvation.

All fibres which die do so either by suicide, starvation,
or blockage. Dead fibres will be reaped by the garbage
collector provided they're unreachable. It is important
for the creator of fibres and their connecting channels
to forget the channels to ensure this occurs.

Unlike typical pre-emptive threading systems, deadlock
is not an error. However a lock up which should lead to
reaping of fibres but which fails to do so because
they remain reachable is univerally an error. This is
known as a livelock: it leads to zombie fibres.

This usually occurs because some other fibre is statically
capable of resolving the lockup, but does not do so 
dynamically. To prevent livelocks, variables holding
channel values to which no I/O will occur dynamically
should also go out of scope.

\chapter{Listings}
\listoflistings 
\end{document}
